\PassOptionsToPackage{unicode=true}{hyperref} % options for packages loaded elsewhere
\PassOptionsToPackage{hyphens}{url}
%
\documentclass[]{article}
\usepackage{lmodern}
\usepackage{amssymb,amsmath}
\usepackage{ifxetex,ifluatex}
\usepackage{fixltx2e} % provides \textsubscript
\ifnum 0\ifxetex 1\fi\ifluatex 1\fi=0 % if pdftex
  \usepackage[T1]{fontenc}
  \usepackage[utf8]{inputenc}
  \usepackage{textcomp} % provides euro and other symbols
\else % if luatex or xelatex
  \usepackage{unicode-math}
  \defaultfontfeatures{Ligatures=TeX,Scale=MatchLowercase}
\fi
% use upquote if available, for straight quotes in verbatim environments
\IfFileExists{upquote.sty}{\usepackage{upquote}}{}
% use microtype if available
\IfFileExists{microtype.sty}{%
\usepackage[]{microtype}
\UseMicrotypeSet[protrusion]{basicmath} % disable protrusion for tt fonts
}{}
\IfFileExists{parskip.sty}{%
\usepackage{parskip}
}{% else
\setlength{\parindent}{0pt}
\setlength{\parskip}{6pt plus 2pt minus 1pt}
}
\usepackage{hyperref}
\hypersetup{
            pdftitle={Epigraphic cleaning models},
            pdfauthor={Petra Hermankova},
            pdfborder={0 0 0},
            breaklinks=true}
\urlstyle{same}  % don't use monospace font for urls
\usepackage[margin=1in]{geometry}
\usepackage{color}
\usepackage{fancyvrb}
\newcommand{\VerbBar}{|}
\newcommand{\VERB}{\Verb[commandchars=\\\{\}]}
\DefineVerbatimEnvironment{Highlighting}{Verbatim}{commandchars=\\\{\}}
% Add ',fontsize=\small' for more characters per line
\usepackage{framed}
\definecolor{shadecolor}{RGB}{248,248,248}
\newenvironment{Shaded}{\begin{snugshade}}{\end{snugshade}}
\newcommand{\AlertTok}[1]{\textcolor[rgb]{0.94,0.16,0.16}{#1}}
\newcommand{\AnnotationTok}[1]{\textcolor[rgb]{0.56,0.35,0.01}{\textbf{\textit{#1}}}}
\newcommand{\AttributeTok}[1]{\textcolor[rgb]{0.77,0.63,0.00}{#1}}
\newcommand{\BaseNTok}[1]{\textcolor[rgb]{0.00,0.00,0.81}{#1}}
\newcommand{\BuiltInTok}[1]{#1}
\newcommand{\CharTok}[1]{\textcolor[rgb]{0.31,0.60,0.02}{#1}}
\newcommand{\CommentTok}[1]{\textcolor[rgb]{0.56,0.35,0.01}{\textit{#1}}}
\newcommand{\CommentVarTok}[1]{\textcolor[rgb]{0.56,0.35,0.01}{\textbf{\textit{#1}}}}
\newcommand{\ConstantTok}[1]{\textcolor[rgb]{0.00,0.00,0.00}{#1}}
\newcommand{\ControlFlowTok}[1]{\textcolor[rgb]{0.13,0.29,0.53}{\textbf{#1}}}
\newcommand{\DataTypeTok}[1]{\textcolor[rgb]{0.13,0.29,0.53}{#1}}
\newcommand{\DecValTok}[1]{\textcolor[rgb]{0.00,0.00,0.81}{#1}}
\newcommand{\DocumentationTok}[1]{\textcolor[rgb]{0.56,0.35,0.01}{\textbf{\textit{#1}}}}
\newcommand{\ErrorTok}[1]{\textcolor[rgb]{0.64,0.00,0.00}{\textbf{#1}}}
\newcommand{\ExtensionTok}[1]{#1}
\newcommand{\FloatTok}[1]{\textcolor[rgb]{0.00,0.00,0.81}{#1}}
\newcommand{\FunctionTok}[1]{\textcolor[rgb]{0.00,0.00,0.00}{#1}}
\newcommand{\ImportTok}[1]{#1}
\newcommand{\InformationTok}[1]{\textcolor[rgb]{0.56,0.35,0.01}{\textbf{\textit{#1}}}}
\newcommand{\KeywordTok}[1]{\textcolor[rgb]{0.13,0.29,0.53}{\textbf{#1}}}
\newcommand{\NormalTok}[1]{#1}
\newcommand{\OperatorTok}[1]{\textcolor[rgb]{0.81,0.36,0.00}{\textbf{#1}}}
\newcommand{\OtherTok}[1]{\textcolor[rgb]{0.56,0.35,0.01}{#1}}
\newcommand{\PreprocessorTok}[1]{\textcolor[rgb]{0.56,0.35,0.01}{\textit{#1}}}
\newcommand{\RegionMarkerTok}[1]{#1}
\newcommand{\SpecialCharTok}[1]{\textcolor[rgb]{0.00,0.00,0.00}{#1}}
\newcommand{\SpecialStringTok}[1]{\textcolor[rgb]{0.31,0.60,0.02}{#1}}
\newcommand{\StringTok}[1]{\textcolor[rgb]{0.31,0.60,0.02}{#1}}
\newcommand{\VariableTok}[1]{\textcolor[rgb]{0.00,0.00,0.00}{#1}}
\newcommand{\VerbatimStringTok}[1]{\textcolor[rgb]{0.31,0.60,0.02}{#1}}
\newcommand{\WarningTok}[1]{\textcolor[rgb]{0.56,0.35,0.01}{\textbf{\textit{#1}}}}
\usepackage{graphicx,grffile}
\makeatletter
\def\maxwidth{\ifdim\Gin@nat@width>\linewidth\linewidth\else\Gin@nat@width\fi}
\def\maxheight{\ifdim\Gin@nat@height>\textheight\textheight\else\Gin@nat@height\fi}
\makeatother
% Scale images if necessary, so that they will not overflow the page
% margins by default, and it is still possible to overwrite the defaults
% using explicit options in \includegraphics[width, height, ...]{}
\setkeys{Gin}{width=\maxwidth,height=\maxheight,keepaspectratio}
\setlength{\emergencystretch}{3em}  % prevent overfull lines
\providecommand{\tightlist}{%
  \setlength{\itemsep}{0pt}\setlength{\parskip}{0pt}}
\setcounter{secnumdepth}{0}
% Redefines (sub)paragraphs to behave more like sections
\ifx\paragraph\undefined\else
\let\oldparagraph\paragraph
\renewcommand{\paragraph}[1]{\oldparagraph{#1}\mbox{}}
\fi
\ifx\subparagraph\undefined\else
\let\oldsubparagraph\subparagraph
\renewcommand{\subparagraph}[1]{\oldsubparagraph{#1}\mbox{}}
\fi

% set default figure placement to htbp
\makeatletter
\def\fps@figure{htbp}
\makeatother


\title{Epigraphic cleaning models}
\author{Petra Hermankova}
\date{05/05/2020}

\begin{document}
\maketitle

\emph{Setting up the environment}

\begin{Shaded}
\begin{Highlighting}[]
\NormalTok{knitr}\OperatorTok{::}\NormalTok{opts_chunk}\OperatorTok{$}\KeywordTok{set}\NormalTok{(}\DataTypeTok{echo =} \OtherTok{TRUE}\NormalTok{, }\DataTypeTok{error=}\OtherTok{TRUE}\NormalTok{)}
\NormalTok{devtools}\OperatorTok{::}\KeywordTok{install_github}\NormalTok{(}\StringTok{"mplex/cedhar"}\NormalTok{, }\DataTypeTok{subdir=}\StringTok{"pkg/sdam"}\NormalTok{)}
\CommentTok{#install.packages("rjson")}
\CommentTok{#install.packages("tidyverse")}
\CommentTok{#install.packages("getPass")}
\CommentTok{#install.packages("formatR")}

\KeywordTok{library}\NormalTok{(tidyverse)}
\KeywordTok{library}\NormalTok{(tidytext)}
\KeywordTok{library}\NormalTok{(dplyr)}
\KeywordTok{library}\NormalTok{(stringr)}
\KeywordTok{library}\NormalTok{(sdam)}
\KeywordTok{library}\NormalTok{(rjson)}
\KeywordTok{library}\NormalTok{(getPass)}
\KeywordTok{library}\NormalTok{(formatR)}
\end{Highlighting}
\end{Shaded}

\hypertarget{cleaning-epigraphic-text-for-tidy-text-mining}{%
\section{Cleaning epigraphic text for tidy text
mining}\label{cleaning-epigraphic-text-for-tidy-text-mining}}

Before we attempt the cleaning itself, we need to build the cleaning
blocks. Once the cleaning blocks are ready we can put them together
based on the desired outcome.

I have created three categores of building blocks, closely linked with
the methodological approach and the purpose of the cleaning process.

\begin{enumerate}
\def\labelenumi{\arabic{enumi}.}
\tightlist
\item
  Conservative model category
\item
  Interpretive model category
\item
  Generic cleaning common for both previous categories
\end{enumerate}

\emph{Structure of a cleaning block:}

Each of the cleaning blocks have the same structure. Regular expressions
will be used to find and replace the searched term or pattern.

\texttt{regexpatternname\ \textless{}-\ c("regexpattern",\ "substitutionpattern")}

\hypertarget{building-blocks-for-the-conservative-model}{%
\subsection{1. Building blocks for the conservative
model}\label{building-blocks-for-the-conservative-model}}

\emph{The aim of this model is to produce a clean text that is as close
to the original text of an inscription as possible.}

The cleaned output of the conservative model will be as close to the
original text as possible. In most cases it should resemble a diplomatic
edition of epigraphic text with spaces between words, lowercase letters,
eliminated brackets and non-utf compliant symbols.

\hypertarget{expanded-abbreviations}{%
\subsubsection{1.1. Expanded
abbreviations}\label{expanded-abbreviations}}

\textbf{Aim:} All expanded abbreaviations that are in the parenthesis ()
will be eliminated from the clean text (substituted with "").

\begin{itemize}
\tightlist
\item
  Example before cleaning: \texttt{Αὐρ(ήλιος)\ Οὐαλέριος}
\item
  Example after cleaning: \texttt{Αὐρ\ Οὐαλέριος}
\end{itemize}

\begin{Shaded}
\begin{Highlighting}[]
\NormalTok{expanded_abbreviations_conservative <-}\StringTok{ }\KeywordTok{c}\NormalTok{(}\StringTok{"}\CharTok{\textbackslash{}\textbackslash{}}\StringTok{([^(]*}\CharTok{\textbackslash{}\textbackslash{}}\StringTok{)"}\NormalTok{, }\StringTok{""}\NormalTok{)}
\end{Highlighting}
\end{Shaded}

\hypertarget{suppresion-of-a-text-with-superscripts}{%
\subsubsection{1.2. Suppresion of a text with
superscripts}\label{suppresion-of-a-text-with-superscripts}}

\textbf{Aim:} All supressions that are in the curly braces \{\} followed
by one or more superscript digits will be eliminated from the clean text
(substituted with "").

\textbf{!!!} It is crutial that block
\texttt{3.\ Supression\ of\ a\ text} does not precede block
\texttt{2.\ Supression\ of\ a\ text\ with\ superscripts}, otherwise the
Regex pattern would not clean the text properly. This particular pattern
is common for the PHI dataset and may or may not appear in other
datasets.

\begin{itemize}
\tightlist
\item
  Example before cleaning:
  \texttt{ἱερεὺς\ ληφθὶς\ ὑπὰ\ \{²⁶ὑπὸ\}²⁶\ τῶν\ βαρβάρων}
\item
  Example after cleaning: \texttt{ἱερεὺς\ ληφθὶς\ ὑπὰ\ \ τῶν\ βαρβάρων}
\end{itemize}

\begin{Shaded}
\begin{Highlighting}[]
\NormalTok{suppresion_superscripts_conservative <-}\StringTok{ }\KeywordTok{c}\NormalTok{(}\StringTok{"\{[^\}]*\}[⁰¹²³⁴⁵⁶⁷⁸⁹]+"}\NormalTok{, }\StringTok{""}\NormalTok{)}
\end{Highlighting}
\end{Shaded}

\hypertarget{suppresion-of-a-text}{%
\subsubsection{1.3. Suppresion of a text}\label{suppresion-of-a-text}}

\textbf{Aim:} All curly braces \{\} will be eliminated from the clean
text (substituted with ""), while the contents of the braces will remain
in the text.

\textbf{!!!} It is crutial that block
\texttt{3.\ Supression\ of\ a\ text} does not precede block
\texttt{2.\ Supression\ of\ a\ text\ with\ superscripts}, otherwise the
Regex pattern would not clean the text properly.

\begin{itemize}
\tightlist
\item
  Example before cleaning:
  \texttt{Σεβαστοῦ\ υἱοῦ\ \{θ̣εοῦ\ Σεβαστοῦ\}\ τύχης}
\item
  Example after cleaning: \texttt{Σεβαστοῦ\ υἱοῦ\ θ̣εοῦ\ Σεβαστοῦ\ τύχης}
\end{itemize}

\begin{Shaded}
\begin{Highlighting}[]
\NormalTok{suppresion_conservative <-}\StringTok{ }\KeywordTok{c}\NormalTok{(}\StringTok{"[}\CharTok{\textbackslash{}\textbackslash{}}\StringTok{\{*}\CharTok{\textbackslash{}\textbackslash{}}\StringTok{\}]"}\NormalTok{, }\StringTok{""}\NormalTok{)}
\end{Highlighting}
\end{Shaded}

\hypertarget{restoration}{%
\subsubsection{1.4. Restoration}\label{restoration}}

\textbf{Aim:} All restoration that are in the square brackets {[}{]}
will be eliminated from the clean text (substituted with "").

\textbf{!!!} Beware that by eliminating the contents of the brackets you
may loose some context - use at your own discretion.

\begin{itemize}
\tightlist
\item
  Example before cleaning:
  \texttt{{[}Ν{]}ανα\ Ἕλληνο̣{[}ς{]}\ θυγάτηρ\ καὶ\ ἡ\ ἑτέρα\ {[}γυνὴ{]}}
\item
  Example after cleaning: \texttt{ανα\ Ἕλληνο\ θυγάτηρ\ καὶ\ ἡ\ ἑτέρα}
\end{itemize}

\begin{Shaded}
\begin{Highlighting}[]
\NormalTok{restoration_conservative <-}\StringTok{ }\KeywordTok{c}\NormalTok{(}\StringTok{"}\CharTok{\textbackslash{}\textbackslash{}}\StringTok{[[^[]*}\CharTok{\textbackslash{}\textbackslash{}}\StringTok{]"}\NormalTok{, }\StringTok{""}\NormalTok{)}
\end{Highlighting}
\end{Shaded}

\hypertarget{substitution}{%
\subsubsection{1.5. Substitution}\label{substitution}}

\textbf{Aim:} All substitutions that are in the angular brackets
\textless{}\textgreater{} will be eliminated from the clean text
(substituted with "").

\textbf{!!!} Beware that by eliminating the contents of the brackets you
may loose some context - use at your own discretion.

\begin{itemize}
\tightlist
\item
  Example before cleaning:
  \texttt{κωρο\textless{}ν\ Ἀ\textgreater{}ντιόχ\textless{}ου\textgreater{}\ ἡ\ πατρὶς\ τειμῆ\textless{}ς\textgreater{}}
\item
  Example after cleaning: \texttt{κωρο\ ντιόχ\ ἡ\ πατρὶς\ τειμῆς}
\end{itemize}

\begin{Shaded}
\begin{Highlighting}[]
\NormalTok{substitution_conservative <-}\StringTok{ }\KeywordTok{c}\NormalTok{(}\StringTok{"}\CharTok{\textbackslash{}\textbackslash{}}\StringTok{<[^<]*}\CharTok{\textbackslash{}\textbackslash{}}\StringTok{>"}\NormalTok{, }\StringTok{""}\NormalTok{)}
\end{Highlighting}
\end{Shaded}

\hypertarget{substitution-in-edh-dataset}{%
\subsubsection{1.6. Substitution in EDH
dataset}\label{substitution-in-edh-dataset}}

\textbf{Aim:} All sustitutions following the pattern ``A=B'' will be
cleaned thw following way: B remain in the text and the equal sign and A
will be eliminated from the clean text.

\textbf{!!!} Beware that by eliminating the brackets you may loose some
information about the preservation of the text - use at your own
discretion. The \texttt{substitution\_edh\_interpretive} should be run
before \texttt{substitution\_interpretive} script, otherwise the Regex
pattern would not clean the text properly. The
\texttt{substitution\_interpretive} script will clean the angular
brackets in the next step.

\begin{itemize}
\tightlist
\item
  Example before cleaning:
  \texttt{pos\textless{}u=I\textgreater{}erunt\ bene\ merenti}
\item
  Example after cleaning:
  \texttt{pos\textless{}I\textgreater{}erunt\ bene\ merenti}
\end{itemize}

\begin{Shaded}
\begin{Highlighting}[]
\NormalTok{substitution_edh_conservative <-}\StringTok{ }\KeywordTok{c}\NormalTok{(}\StringTok{"([α-ωΑ-Ωa-zA-Z])=([α-ωΑ-Ωa-zA-Z])"}\NormalTok{, }\StringTok{"}\CharTok{\textbackslash{}\textbackslash{}}\StringTok{2"}\NormalTok{)}
\end{Highlighting}
\end{Shaded}

\hypertarget{building-blocks-for-the-interpretive-model}{%
\subsection{2. Building blocks for the interpretive
model}\label{building-blocks-for-the-interpretive-model}}

\emph{The aim of this model is to produce a clean text that is enriched
with editorial interpretations of the original text.}

The output of the interpretive model will produce an epigraphic text
with as many editorial suggestions, restorations, corrections, and
improvements as possible to provide as much possible contents of the
inscription as possible. The brackets and non-utf compliant symbols will
be eliminated.

\hypertarget{expanded-abbreviations-1}{%
\subsubsection{2.1. Expanded
abbreviations}\label{expanded-abbreviations-1}}

\textbf{Aim:} All parenthesis () will be eliminated from the clean text
(substituted with ""), while the contents of the parenthesis will remain
in the text.

\begin{itemize}
\tightlist
\item
  Example before cleaning: \texttt{Αὐρ(ήλιος)\ Οὐαλέριος}
\item
  Example after cleaning: \texttt{Αὐρήλιος\ Οὐαλέριος}
\end{itemize}

\begin{Shaded}
\begin{Highlighting}[]
\NormalTok{expanded_abbreviations_interpretive <-}\StringTok{ }\KeywordTok{c}\NormalTok{(}\StringTok{"[}\CharTok{\textbackslash{}\textbackslash{}}\StringTok{(*}\CharTok{\textbackslash{}\textbackslash{}}\StringTok{)]"}\NormalTok{, }\StringTok{""}\NormalTok{)}
\end{Highlighting}
\end{Shaded}

\hypertarget{suppresion-of-a-text-with-superscripts-1}{%
\subsubsection{2.2. Suppresion of a text with
superscripts}\label{suppresion-of-a-text-with-superscripts-1}}

\textbf{Aim:} Contents found within curly braces \{\} followed by one or
more superscript digits will substitute the word immediately preceding
the curly braces, see example.

\textbf{!!!} It is crutial that block
\texttt{3.\ Supression\ of\ a\ text} does not precede block
\texttt{2.\ Supression\ of\ a\ text\ with\ superscripts}, otherwise the
Regex pattern would not clean the text properly. This particular pattern
is common for the PHI dataset and may or may not appear in other
datasets.

\begin{itemize}
\tightlist
\item
  Example before cleaning:
  \texttt{ἱερεὺς\ ληφθὶς\ ὑπὰ\ \{²⁶ὑπὸ\}²⁶\ τῶν\ βαρβάρων}
\item
  Example after cleaning: \texttt{ἱερεὺς\ ληφθὶς\ ὑπὸ\ τῶν\ βαρβάρων}
\end{itemize}

\begin{Shaded}
\begin{Highlighting}[]
\NormalTok{suppresion_superscripts_interpretive <-}\StringTok{ }\KeywordTok{c}\NormalTok{(}\StringTok{" [^ ]+ }\CharTok{\textbackslash{}\textbackslash{}}\StringTok{\{([⁰¹²³⁴⁵⁶⁷⁸⁹]+)([^\}]+)}\CharTok{\textbackslash{}\textbackslash{}}\StringTok{\}}\CharTok{\textbackslash{}\textbackslash{}}\StringTok{1"}\NormalTok{, }\StringTok{" }\CharTok{\textbackslash{}\textbackslash{}}\StringTok{2"}\NormalTok{)}
\end{Highlighting}
\end{Shaded}

\emph{Note:} the script will not work if there is no text preceeding the
curly braces. To eliminate the curly braces with superscripts and the
contents of the curly braces, use the
\texttt{suppresion\_superscripts\_conservative} script. However, it is
recommended to run the \texttt{suppresion\_superscripts\_conservative}
script after \texttt{suppresion\_superscripts\_interpretive} script,
otherwise the Regex pattern would not clean the text properly.

\hypertarget{suppresion-of-a-text-1}{%
\subsubsection{2.3. Suppresion of a text}\label{suppresion-of-a-text-1}}

\textbf{Aim:} All curly braces \{\} will be eliminated from the clean
text (substituted with ""), while the contents of the braces will remain
in the text.

\textbf{!!!} It is crutial that block
\texttt{3.\ Supression\ of\ a\ text} does not precede block
\texttt{2.\ Supression\ of\ a\ text\ with\ superscripts}, otherwise the
Regex pattern would not clean the text properly. Due to ambiguous use of
\{\} by editors of epigraphic corpora, the exact usage depends on the
specific dataset and the way the curly braces were used. If you wish to
keep the text within the brackets, use
\texttt{suppresion\_keep\_interpretive} script and if you wish to remove
the text in the brackets, use \texttt{suppresion\_remove\_interpretive}
script.

\begin{itemize}
\tightlist
\item
  Example before cleaning:
  \texttt{θ̣εοῦ\ Σεβαστοῦ\ υἱοῦ\ \{θ̣εοῦ\ Σεβαστοῦ\}\ τύχης}
\item
  Example after cleaning (keep text):
  \texttt{θ̣εοῦ\ Σεβαστοῦ\ υἱοῦ\ θ̣εοῦ\ Σεβαστοῦ\ τύχης}
\item
  Example after cleaning (remove text):
  \texttt{θ̣εοῦ\ Σεβαστοῦ\ υἱοῦ\ \ τύχης}
\end{itemize}

\begin{Shaded}
\begin{Highlighting}[]
\NormalTok{suppresion_keep_interpretive <-}\StringTok{ }\KeywordTok{c}\NormalTok{(}\StringTok{"[}\CharTok{\textbackslash{}\textbackslash{}}\StringTok{\{*}\CharTok{\textbackslash{}\textbackslash{}}\StringTok{\}]"}\NormalTok{, }\StringTok{""}\NormalTok{)}
\end{Highlighting}
\end{Shaded}

OR

\begin{Shaded}
\begin{Highlighting}[]
\NormalTok{suppresion_remove_interpretive <-}\StringTok{ }\KeywordTok{c}\NormalTok{(}\StringTok{"\{[^\}]*\}"}\NormalTok{, }\StringTok{""}\NormalTok{)}
\end{Highlighting}
\end{Shaded}

\hypertarget{restoration-1}{%
\subsubsection{2.4. Restoration}\label{restoration-1}}

\textbf{Aim:} All square brackets {[}{]} will be eliminated from the
clean text (substituted with ""), while the contents of the brackets
will remain in the text.

\textbf{!!!} Beware that by eliminating the brackets you may loose some
information about the preservation of the text - use at your own
discretion.

\begin{itemize}
\tightlist
\item
  Example before cleaning:
  \texttt{{[}Ν{]}ανα\ Ἕλληνο̣{[}ς{]}\ θυγάτηρ\ καὶ\ ἡ\ ἑτέρα\ {[}γυνὴ{]}}
\item
  Example after cleaning:
  \texttt{Νανα\ Ἕλληνο̣ς\ θυγάτηρ\ καὶ\ ἡ\ ἑτέρα\ γυνὴ}
\end{itemize}

\begin{Shaded}
\begin{Highlighting}[]
\NormalTok{restoration_interpretive <-}\StringTok{ }\KeywordTok{c}\NormalTok{(}\StringTok{"[}\CharTok{\textbackslash{}\textbackslash{}}\StringTok{[*}\CharTok{\textbackslash{}\textbackslash{}}\StringTok{]]"}\NormalTok{, }\StringTok{""}\NormalTok{)}
\end{Highlighting}
\end{Shaded}

\hypertarget{substitution-1}{%
\subsubsection{2.5. Substitution}\label{substitution-1}}

\textbf{Aim:} All angular brackets \textless{}\textgreater{} will be
eliminated from the clean text (substituted with ""), while the contents
of the brackets will remain in the text.

\textbf{!!!} Beware that by eliminating the brackets you may loose some
information about the preservation of the text - use at your own
discretion.

\begin{itemize}
\tightlist
\item
  Example before cleaning:
  \texttt{κωρο\textless{}ν\ Ἀ\textgreater{}ντιόχ\textless{}ου\textgreater{}\ ἡ\ πατρὶς\ τειμῆ\textless{}ς\textgreater{}}
\item
  Example after cleaning: \texttt{κωρον\ Ἀντιόχου\ ἡ\ πατρὶς\ τειμῆς}
\end{itemize}

\begin{Shaded}
\begin{Highlighting}[]
\NormalTok{substitution_interpretive <-}\StringTok{ }\KeywordTok{c}\NormalTok{(}\StringTok{"[}\CharTok{\textbackslash{}\textbackslash{}}\StringTok{<*}\CharTok{\textbackslash{}\textbackslash{}}\StringTok{>]"}\NormalTok{, }\StringTok{""}\NormalTok{)}
\end{Highlighting}
\end{Shaded}

\hypertarget{substitution-in-edh-dataset-1}{%
\subsubsection{2.6. Substitution in EDH
dataset}\label{substitution-in-edh-dataset-1}}

\textbf{Aim:} All sustitutions following the pattern ``A=B'' will be
cleaned thw following way: A remain in the text and the equal sign and B
will be eliminated from the clean text.

\textbf{!!!} Beware that by eliminating the brackets you may loose some
information about the preservation of the text - use at your own
discretion. The \texttt{substitution\_edh\_interpretive} should be run
before \texttt{substitution\_interpretive} script, otherwise the Regex
pattern would not clean the text properly. The
\texttt{substitution\_interpretive} script will clean the angular
brackets in the next step.

\begin{itemize}
\tightlist
\item
  Example before cleaning:
  \texttt{pos\textless{}u=I\textgreater{}erunt\ bene\ merenti}
\item
  Example after cleaning:
  \texttt{pos\textless{}u\textgreater{}erunt\ bene\ merenti}
\end{itemize}

\begin{Shaded}
\begin{Highlighting}[]
\NormalTok{substitution_edh_interpretive <-}\StringTok{ }\KeywordTok{c}\NormalTok{(}\StringTok{"([α-ωΑ-Ωa-zA-Z])=([α-ωΑ-Ωa-zA-Z])"}\NormalTok{, }\StringTok{"}\CharTok{\textbackslash{}\textbackslash{}}\StringTok{1"}\NormalTok{)}
\end{Highlighting}
\end{Shaded}

\hypertarget{the-generic-text-cleaning}{%
\subsection{3. The generic text
cleaning}\label{the-generic-text-cleaning}}

\emph{The aim of the generic cleaning is to strip the epigraphic text
any non-utf compliant symbols and characters that do not adhere to the
principles of a `tidy text' analysis.}

The final output of the cleaning depends on which of the individual
cleaning blocks will be in the cleaning script. Each individual block
represents one step of the cleaning process, and user can modify all the
steps to recah the intended outcome. All the cleaning steps are
dependent on the characteristics of the original dataset, therefore
familiarity with the original dataset prior the cleaning process is
recommended. Each dataset can have a different set of symbols and
characters to be cleaned, thus, the cleaning blocks should be adjusted
accordingly.

\hypertarget{lacuna-1}{%
\subsubsection{3.1. Lacuna 1}\label{lacuna-1}}

\textbf{Aim:} All square brackets {[}{]} containing one or more ``---''
will be eliminated from the clean text (substituted with "").

\textbf{!!!} The scipt \texttt{lacuna1} should be run before
\texttt{restoration\_conservative} and
\texttt{restoration\_interpretive} scripts, otherwise the Regex pattern
would not clean the text properly.

\begin{itemize}
\tightlist
\item
  Example before cleaning: \texttt{{[}—\ —\ —{]}ης\ θεῷ\ Φοίβῳ}
\item
  Example after cleaning: \texttt{ης\ θεῷ\ Φοίβῳ}
\end{itemize}

\begin{Shaded}
\begin{Highlighting}[]
\NormalTok{lacuna1 <-}\StringTok{ }\KeywordTok{c}\NormalTok{(}\StringTok{"}\CharTok{\textbackslash{}\textbackslash{}}\StringTok{[[— ]+}\CharTok{\textbackslash{}\textbackslash{}}\StringTok{]"}\NormalTok{, }\StringTok{""}\NormalTok{)}
\end{Highlighting}
\end{Shaded}

\emph{Note:} If there is a text within the square bracket, e.g.
\texttt{προύχον{[}τος\ —\ —\ —{]}}, script
\texttt{restoration\_interpretive} will eliminate the square brackets,
the script \texttt{interpunction\_symbols} will clean the ``---'' and
the script \texttt{multi\_whitespace} will eliminate the extra
whitespaces. Therefore the scripts
\texttt{restoration\_interpretive}(1),
\texttt{interpunction\_symbols}(2) and \texttt{multi\_whitespace}(3)
should be used in combination and in the indicated sequence.

\hypertarget{lacuna-2}{%
\subsubsection{3.2. Lacuna 2}\label{lacuna-2}}

\textbf{Aim:} All square brackets {[}{]} containing one or more ``.''
will be eliminated from the clean text (substituted with "").

\textbf{!!!} The scipt \texttt{lacuna1} should be run before
\texttt{restoration\_conservative} and
\texttt{restoration\_interpretive} scripts, otherwise the Regex pattern
would not clean the text properly.

\begin{itemize}
\tightlist
\item
  Example before cleaning: \texttt{{[}․․{]}ω\ Διὶ\ καὶ\ Ἥρᾳ}
\item
  Example after cleaning: \texttt{ω\ Διὶ\ καὶ\ Ἥρᾳ}
\end{itemize}

\begin{Shaded}
\begin{Highlighting}[]
\NormalTok{lacuna2 <-}\StringTok{ }\KeywordTok{c}\NormalTok{(}\StringTok{"}\CharTok{\textbackslash{}\textbackslash{}}\StringTok{[[․]+}\CharTok{\textbackslash{}\textbackslash{}}\StringTok{]"}\NormalTok{, }\StringTok{""}\NormalTok{)}
\end{Highlighting}
\end{Shaded}

\emph{Note:} If there is a text within the square bracket, e.g.
\texttt{προύχον{[}τος\ —\ —\ —{]}}, script
\texttt{restoration\_interpretive} will eliminate the square brackets,
the script \texttt{interpunction\_symbols} will clean the ``---'' and
the script \texttt{multi\_whitespace} will eliminate the extra
whitespaces. Therefore the scripts
\texttt{restoration\_interpretive}(1),
\texttt{interpunction\_symbols}(2) and \texttt{multi\_whitespace}(3)
should be used in combination and in the indicated sequence.

\hypertarget{vacat}{%
\subsubsection{3.3. Vacat}\label{vacat}}

\textbf{Aim:} All instances of the following strings ``vacat, vac, vac.,
v.'' will be replaced by a space (substituted with " "). If there is any
extra whitespace, it will be cleaned by \texttt{multi\_whitespace}
script in the following steps.

\textbf{!!!} If your datasets contains latin inscriptions, you may want
to check whether the \texttt{vacat} script is not eliminitating more
words than anticipated, e.g.~words containing string ``vacat'' or
``vac''. If so, adjust the cleaning block accordingly, i.e.~remove
``vac'', or don't use it.

\begin{itemize}
\tightlist
\item
  Example before cleaning: \texttt{Ἡρακλείδα\ vacat\ χαῖρε.}
\item
  Example after cleaning: \texttt{Ἡρακλείδα\ \ \ \ χαῖρε.}
\end{itemize}

\begin{Shaded}
\begin{Highlighting}[]
\NormalTok{vacat <-}\StringTok{ }\KeywordTok{c}\NormalTok{(}\StringTok{"(vacat|vac|vac}\CharTok{\textbackslash{}\textbackslash{}}\StringTok{.|v}\CharTok{\textbackslash{}\textbackslash{}}\StringTok{.)"}\NormalTok{, }\StringTok{" "}\NormalTok{)}
\end{Highlighting}
\end{Shaded}

\hypertarget{editorial-notes}{%
\subsubsection{3.4. Editorial notes}\label{editorial-notes}}

\textbf{Aim:} All instances of the editorial strings in parenthesis such
as (vel sim.) will be replaced by a space (substituted with " "). If
there is any extra whitespace, it will be cleaned by
\texttt{multi\_whitespace} script in the following steps.

\textbf{!!!} The \texttt{editorial\_notes} script should run before the
\texttt{expanded\_abbreviations\_conservative} and
\texttt{expanded\_abbreviations\_interpretive} scripts, otherwise the
Regex pattern would not clean the text properly.

\begin{itemize}
\tightlist
\item
  Example before cleaning: \texttt{Ἥρωι\ (vel\ sim.)\ Καλλισθένης}
\item
  Example after cleaning: ``Ἥρωι Καλλισθένης```
\end{itemize}

\begin{Shaded}
\begin{Highlighting}[]
\NormalTok{editorial_notes <-}\KeywordTok{c}\NormalTok{(}\StringTok{"}\CharTok{\textbackslash{}\textbackslash{}}\StringTok{(vel sim.}\CharTok{\textbackslash{}\textbackslash{}}\StringTok{)"}\NormalTok{, }\StringTok{" "}\NormalTok{)}
\end{Highlighting}
\end{Shaded}

\hypertarget{new-line}{%
\subsubsection{3.5. New line}\label{new-line}}

\textbf{Aim:} All instances of in-line symbol for new line (\textbar{})
will be eliminated (substituted with "").

\begin{itemize}
\tightlist
\item
  Example before cleaning: \texttt{Λάμπρη\ Τ̣ελεσήνορ\textbar{}ος\ γυνή.}
\item
  Example after cleaning: \texttt{Λάμπρη\ Τ̣ελεσήνορος\ γυνή}
\end{itemize}

\begin{Shaded}
\begin{Highlighting}[]
\NormalTok{new_line <-}\StringTok{ }\KeywordTok{c}\NormalTok{(}\StringTok{"[}\CharTok{\textbackslash{}\textbackslash{}}\StringTok{||}\CharTok{\textbackslash{}\textbackslash{}}\StringTok{/]"}\NormalTok{, }\StringTok{""}\NormalTok{)}
\end{Highlighting}
\end{Shaded}

\hypertarget{split-word-over-two-lines}{%
\subsubsection{3.6. Split word over two
lines}\label{split-word-over-two-lines}}

\textbf{Aim:} All instances of words split between two lines with a dash
(-) will be eliminated (substituted with "").

\begin{itemize}
\tightlist
\item
  Example before cleaning:
  \texttt{ἀρχιερέως\ καὶ\ εὐποσιάρ-\textbackslash{}nχου\ μηνὸς}
\item
  Example after cleaning: \texttt{ἀρχιερέως\ καὶ\ εὐποσιάρχου\ μηνὸς}
\end{itemize}

\begin{Shaded}
\begin{Highlighting}[]
\NormalTok{split_word_multiline <-}\StringTok{ }\KeywordTok{c}\NormalTok{(}\StringTok{"-}\CharTok{\textbackslash{}\textbackslash{}}\StringTok{n"}\NormalTok{, }\StringTok{""}\NormalTok{)}
\end{Highlighting}
\end{Shaded}

\hypertarget{erasure-empty}{%
\subsubsection{3.7. Erasure empty}\label{erasure-empty}}

\textbf{Aim:} All instances of erased text (〚---〛) will be replaced by
a space (substituted with " "). If there is any extra whitespace, it
will be cleaned by \texttt{multi\_whitespace} script in the following
steps.

\begin{itemize}
\tightlist
\item
  Example before cleaning: \texttt{Ἀρτέμιδι\ 〚—\ —\ —〛\ ἐπηκόοις.}
\item
  Example after cleaning: \texttt{Ἀρτέμιδι\ \ ἐπηκόοις.}
\end{itemize}

\begin{Shaded}
\begin{Highlighting}[]
\NormalTok{erasure_empty <-}\StringTok{ }\KeywordTok{c}\NormalTok{(}\StringTok{"〚[— ]+〛"}\NormalTok{, }\StringTok{" "}\NormalTok{)}
\end{Highlighting}
\end{Shaded}

\hypertarget{erasure-with-new-text}{%
\subsubsection{3.8. Erasure with new text}\label{erasure-with-new-text}}

\textbf{Aim:} All instances of double brackets for erasures (〚 〛) will
be eliminated (substituted with "") and the contents of the double
brackets will be preserved as part of the clean text.

\begin{itemize}
\tightlist
\item
  Example before cleaning:
  \texttt{Ἀμύντωρ\ Νουμηνίου\ 〚χαῖρε〛.\ καὶ\ ἡ\ γυνὴ\ αὐτοῦ}
\item
  Example after cleaning:
  \texttt{Ἀμύντωρ\ Νουμηνίου\ χαῖρε.\ καὶ\ ἡ\ γυνὴ\ αὐτοῦ}
\end{itemize}

\begin{Shaded}
\begin{Highlighting}[]
\NormalTok{erasure_new_text <-}\StringTok{ }\KeywordTok{c}\NormalTok{(}\StringTok{"[〚〛]"}\NormalTok{, }\StringTok{""}\NormalTok{)}
\end{Highlighting}
\end{Shaded}

\hypertarget{dubious-dot-subscript}{%
\subsubsection{3.9. Dubious dot subscript}\label{dubious-dot-subscript}}

\textbf{Aim:} All instances of the dubious reading marked by the
subscrit dot (unicode 0323) will be eliminated (substituted with "").

\textbf{!!!} The \texttt{dubious\_dot\_subscript} script should happen
as first step of the cleaning, otherwise the letters might shift and the
Regex pattern would not clean the text properly.

\begin{itemize}
\tightlist
\item
  Example before cleaning: \texttt{Ἀ̣πό̣λ̣λ̣ωνος}
\item
  Example after cleaning: \texttt{Ἀπόλλωνος}
\end{itemize}

\begin{Shaded}
\begin{Highlighting}[]
\NormalTok{dubious_dot_subscript <-}\StringTok{ }\KeywordTok{c}\NormalTok{(}\StringTok{"\textbackslash{}u\{0323\}"}\NormalTok{, }\StringTok{""}\NormalTok{)}
\end{Highlighting}
\end{Shaded}

\hypertarget{interpunction-symbols}{%
\subsubsection{3.10. Interpunction
symbols}\label{interpunction-symbols}}

\textbf{Aim:} All instances of listed interpunction symbols
(,.!----\#\%\^{}\&*/\textasciitilde{}:;) will be replaced by a space
(substituted with " "). If there is any extra whitespace, it will be
cleaned by \texttt{multi\_whitespace} script in the following steps.

\begin{itemize}
\tightlist
\item
  Example before cleaning: \texttt{Φιλήτη\ \#\ θεᾷ\ Μαλοφόρῳ} or
  \texttt{κεῖμαι\ πρόμοιρος\ Ἑρμογένης\ τυμβευθείς·\ /ἀγὼν}
\item
  Example after cleaning: \texttt{Φιλήτη\ \ θεᾷ\ Μαλοφόρῳ} or
  \texttt{κεῖμαι\ πρόμοιρος\ Ἑρμογένης\ τυμβευθείς\ \ \ ἀγὼν}
\end{itemize}

\begin{Shaded}
\begin{Highlighting}[]
\NormalTok{interpunction_symbols <-}\StringTok{ }\KeywordTok{c}\NormalTok{(}\StringTok{"[,|}\CharTok{\textbackslash{}\textbackslash{}}\StringTok{.|․|·|!|}\CharTok{\textbackslash{}\textbackslash{}}\StringTok{-|—|–|#|%|}\CharTok{\textbackslash{}\textbackslash{}}\StringTok{^|&|}\CharTok{\textbackslash{}\textbackslash{}}\StringTok{*|~|:|;|@]"}\NormalTok{, }\StringTok{" "}\NormalTok{)}
\end{Highlighting}
\end{Shaded}

\hypertarget{superscript-numbers}{%
\subsubsection{3.11. Superscript numbers}\label{superscript-numbers}}

\textbf{Aim:} All instances of superscripted numbers will be eliminated
(substituted with "").

\textbf{!!!} The \texttt{superscript\_numbers} should not be run before
the \texttt{suppresion\_superscripts\_conservative} or
\texttt{suppresion\_superscripts\_interpretive} script, otherwise the
Regex pattern would not clean the text properly.

\begin{itemize}
\tightlist
\item
  Example before cleaning:
  \texttt{Αὐρ(ήλιος)\ Διονύσιος\ \#⁵⁶\ βʹ\ \#⁵⁶}
\item
  Example after cleaning: \texttt{Αὐρ(ήλιος)\ Διονύσιος\ \#\ βʹ\ \#}
\end{itemize}

\begin{Shaded}
\begin{Highlighting}[]
\NormalTok{superscript_numbers <-}\StringTok{ }\KeywordTok{c}\NormalTok{(}\StringTok{"[⁰¹²³⁴⁵⁶⁷⁸⁹]+"}\NormalTok{, }\StringTok{""}\NormalTok{)}
\end{Highlighting}
\end{Shaded}

\hypertarget{epigraphic-symbols}{%
\subsubsection{3.12. Epigraphic symbols}\label{epigraphic-symbols}}

\textbf{Aim:} All instances of the listed specialised epigraphic
symbols, such as the haedera (❦), will be eliminated (substituted with
"").

\begin{itemize}
\tightlist
\item
  Example before cleaning: \texttt{ἀγαθῆι\ ❦\ τύχηι}
\item
  Example after cleaning: \texttt{ἀγαθῆι\ \ \ τύχηι}
\end{itemize}

\begin{Shaded}
\begin{Highlighting}[]
\NormalTok{epigraphic_symbols <-}\KeywordTok{c}\NormalTok{ (}\StringTok{"[❦|∙|𐆖|⏑|⏓|⏕]"}\NormalTok{, }\StringTok{""}\NormalTok{)}
\end{Highlighting}
\end{Shaded}

\hypertarget{uncertainty-symbols}{%
\subsubsection{3.13. Uncertainty symbols}\label{uncertainty-symbols}}

\textbf{Aim:} All instances of th elisted symbols marking uncertainty
(?) will be replaced by a space (substituted with " "). If there is any
extra whitespace, it will be cleaned by \texttt{multi\_whitespace}
script in the following steps.

\begin{itemize}
\tightlist
\item
  Example before cleaning: \texttt{χαῖρε?}
\item
  Example after cleaning: \texttt{χαῖρε}
\end{itemize}

\begin{Shaded}
\begin{Highlighting}[]
\NormalTok{uncertainty_symbols <-}\KeywordTok{c}\NormalTok{ (}\StringTok{"[?]"}\NormalTok{, }\StringTok{" "}\NormalTok{)}
\end{Highlighting}
\end{Shaded}

\hypertarget{end-of-line}{%
\subsubsection{3.14. End of line}\label{end-of-line}}

\textbf{Aim:} All instances of end of line symbol (\n) will be replaced
by space (substituted with " ").

\begin{itemize}
\tightlist
\item
  Example before cleaning:
  \texttt{καὶ\ ἄρξαντα\textbackslash{}nτοῦ\ κοινοῦ}
\item
  Example after cleaning: \texttt{καὶ\ ἄρξαντα\ τοῦ\ κοινοῦ}
\end{itemize}

\begin{Shaded}
\begin{Highlighting}[]
\NormalTok{end_line <-}\StringTok{ }\KeywordTok{c}\NormalTok{(}\StringTok{"}\CharTok{\textbackslash{}\textbackslash{}}\StringTok{n"}\NormalTok{, }\StringTok{" "}\NormalTok{)}
\end{Highlighting}
\end{Shaded}

\hypertarget{extra-blank-space}{%
\subsubsection{3.15. Extra blank space}\label{extra-blank-space}}

\textbf{Aim:} All instances of extra blank space (``  '') will be
replaced by space (substituted with " ").

\begin{itemize}
\tightlist
\item
  Example before cleaning: \texttt{ἀγαθῆι  \ τύχηι.}
\item
  Example after cleaning: \texttt{ἀγαθῆι\ τύχηι.}
\end{itemize}

\begin{Shaded}
\begin{Highlighting}[]
\NormalTok{extra_blank <-}\StringTok{ }\KeywordTok{c}\NormalTok{(}\StringTok{"[ ]+"}\NormalTok{, }\StringTok{" "}\NormalTok{)}
\end{Highlighting}
\end{Shaded}

\hypertarget{multi-whitespace}{%
\subsubsection{3.16. Multi-whitespace}\label{multi-whitespace}}

\textbf{Aim:} All instances of more then one whitespace " " next to each
other will be eliminated (substituted with "").

\textbf{!!!} The \texttt{multi\_whitespace} should run as the second
last cleaning block to ensure all redundant white spaces are cleaned
from the text.

\begin{itemize}
\tightlist
\item
  Example before cleaning: \texttt{Ἡρακλείδα\ \ \ \ χαῖρε.}
\item
  Example after cleaning: \texttt{Ἡρακλείδα\ χαῖρε.}
\end{itemize}

\begin{Shaded}
\begin{Highlighting}[]
\NormalTok{multi_whitespace <-}\StringTok{ }\KeywordTok{c}\NormalTok{(}\StringTok{"}\CharTok{\textbackslash{}\textbackslash{}}\StringTok{s+"}\NormalTok{, }\StringTok{" "}\NormalTok{)}
\end{Highlighting}
\end{Shaded}

\hypertarget{trailing-and-leading-whitespace}{%
\subsubsection{3.17. Trailing and leading
whitespace}\label{trailing-and-leading-whitespace}}

\textbf{Aim:} All instances of whitespace " " at the beginning and end
of the line will be eliminated (substituted with "").

\textbf{!!!} The \texttt{whitespace\_endline} should run as the last
cleaning block to ensure all redundant white spaces are cleaned from the
text.

\begin{itemize}
\tightlist
\item
  Example before cleaning: \texttt{χαῖρε}
\item
  Example after cleaning: \texttt{χαῖρε}
\end{itemize}

\begin{Shaded}
\begin{Highlighting}[]
\NormalTok{whitespace_endline <-}\StringTok{ }\KeywordTok{c}\NormalTok{(}\StringTok{"(^}\CharTok{\textbackslash{}\textbackslash{}}\StringTok{s|}\CharTok{\textbackslash{}\textbackslash{}}\StringTok{s$)"}\NormalTok{, }\StringTok{""}\NormalTok{)}
\end{Highlighting}
\end{Shaded}

\hypertarget{editorial-comments-in-latin-alphabet}{%
\subsubsection{3.18. Editorial comments in Latin
alphabet}\label{editorial-comments-in-latin-alphabet}}

\textbf{Aim:} All instances of editorial comments in Latin alphabet that
are enclosed in curly braces \{\} with superscript numbers will be
eliminated (substituted with "").

\textbf{!!!} If your dataset contains Latin inscriptions, use this
script with caution. Verify first, that running the script it does not
eliminate any necessary information or text. This block has been
specifically designed for the interpretive cleaning of the PHI Greek
Inscription dataset and it should run before
\texttt{suppresion\_superscripts\_interpretive} and
\texttt{suppresion\_interpretive} scripts, otherwise the Regex pattern
would not clean the text properly.

\begin{itemize}
\tightlist
\item
  Example before cleaning:
  \texttt{ἀγαθῆι\ τύχηι.\ \{²in\ parte\ inferiore\ altera\ manu\ incisa\ est:\}²\ ὑπὲρ\ τῆς\ τοῦ}
\item
  Example after cleaning: \texttt{ἀγαθῆι\ τύχηι.\ ὑπὲρ\ τῆς\ τοῦ}
\end{itemize}

\begin{Shaded}
\begin{Highlighting}[]
\NormalTok{editorial_comments_latin <-}\StringTok{ }\KeywordTok{c}\NormalTok{(}\StringTok{"}\CharTok{\textbackslash{}\textbackslash{}}\StringTok{\{([⁰¹²³⁴⁵⁶⁷⁸⁹]+)([a-zA-Z0-9][^\}]+)}\CharTok{\textbackslash{}\textbackslash{}}\StringTok{\}}\CharTok{\textbackslash{}\textbackslash{}}\StringTok{1"}\NormalTok{, }\StringTok{""}\NormalTok{)}
\end{Highlighting}
\end{Shaded}

\hypertarget{arabic-numerals}{%
\subsubsection{3.19. Arabic numerals}\label{arabic-numerals}}

\textbf{Aim:} All instances of arabic numerals (0-9) will be eliminated
(substituted with "").

\textbf{!!!} If your dataset contains arabic numerals that you would
like to keep, use this script with caution. Verify first, that running
the script it does not eliminate any necessary information or text. This
block has been specifically designed for the interpretive cleaning of
the PHI Greek Inscription dataset and it should run before
\texttt{multi\_whitespace} and \texttt{whitespace\_endline} scripts,
otherwise the Regex pattern would not clean the text properly.

\begin{itemize}
\tightlist
\item
  Example before cleaning: \texttt{ἡ\ γυνὴ\ αὐτοῦ\ ΦιλΙ̣\ 4\ 5\ καὶ}
\item
  Example after cleaning: \texttt{ἡ\ γυνὴ\ αὐτοῦ\ ΦιλΙ\ καὶ}
\end{itemize}

\begin{Shaded}
\begin{Highlighting}[]
\NormalTok{arabic_numerals <-}\StringTok{ }\KeywordTok{c}\NormalTok{(}\StringTok{"[0-9]+"}\NormalTok{, }\StringTok{""}\NormalTok{)}
\end{Highlighting}
\end{Shaded}

\hypertarget{unclosed-brackets}{%
\subsubsection{3.20 Unclosed brackets}\label{unclosed-brackets}}

\textbf{Aim:} All instances of unclosed brackets will be eliminated
(substituted with "").

\textbf{!!!} Use the \texttt{unclosed\_brackets} script immediately
before \texttt{multi\_whitespace} and \texttt{whitespace\_endline}
scripts, otherwise the Regex pattern would not clean the text properly.

\begin{itemize}
\tightlist
\item
  Example before cleaning: \texttt{ummio\ isenna\ Xv\ {[}}
\item
  Example after cleaning: \texttt{ummio\ isenna\ Xv}
\end{itemize}

\begin{Shaded}
\begin{Highlighting}[]
\NormalTok{unclosed_brackets <-}\StringTok{ }\KeywordTok{c}\NormalTok{(}\StringTok{"[}\CharTok{\textbackslash{}\textbackslash{}}\StringTok{[|}\CharTok{\textbackslash{}\textbackslash{}}\StringTok{\{|}\CharTok{\textbackslash{}\textbackslash{}}\StringTok{(|}\CharTok{\textbackslash{}\textbackslash{}}\StringTok{)|}\CharTok{\textbackslash{}\textbackslash{}}\StringTok{\}|}\CharTok{\textbackslash{}\textbackslash{}}\StringTok{]]"}\NormalTok{, }\StringTok{""}\NormalTok{)}
\end{Highlighting}
\end{Shaded}

\begin{center}\rule{0.5\linewidth}{0.5pt}\end{center}

\hypertarget{building-cleaning-functions-for-specific-datasets}{%
\section{Building cleaning functions for specific
datasets}\label{building-cleaning-functions-for-specific-datasets}}

When we have established the individual buidling blocks, we can put them
together in the right sequence and build a cleaning function in R for
conservative and interpretive models.

\hypertarget{phi-greek-inscriptions-dataset}{%
\subsection{PHI Greek Inscriptions
dataset}\label{phi-greek-inscriptions-dataset}}

Source: \url{https://epigraphy.packhum.org/}

\hypertarget{loading-data}{%
\subsubsection{Loading data}\label{loading-data}}

First, we need to load the provided test dataset
\texttt{PHI\_IGBulg-I.csv} located in the \texttt{test\_data} folder and
create an object \texttt{dirtytext} contain the text to be cleaned. Use
\texttt{getwd()} function to make sure you are in the right working
directory, so the \texttt{read\_csv} code works for you. If not, adjust
the path.

\begin{Shaded}
\begin{Highlighting}[]
\KeywordTok{getwd}\NormalTok{()}
\end{Highlighting}
\end{Shaded}

\begin{verbatim}
## [1] "/home/petra/Github/epigraphic_cleaning/scripts/R"
\end{verbatim}

\begin{Shaded}
\begin{Highlighting}[]
\NormalTok{text <-}\StringTok{ }\KeywordTok{read_csv}\NormalTok{(}\StringTok{"../../test_data/PHI_IGBulg-I.csv"}\NormalTok{)}
\NormalTok{dirtytext <-}\StringTok{ }\KeywordTok{as.data.frame}\NormalTok{(}\KeywordTok{select}\NormalTok{(text, hdr2, data))}
\end{Highlighting}
\end{Shaded}

\hypertarget{conservative-model}{%
\subsubsection{Conservative model}\label{conservative-model}}

\emph{Aim:} to have a clean text that is as close to the original
inscription as preserved on the medium.

\begin{Shaded}
\begin{Highlighting}[]
\NormalTok{cleaning_conservative <-}\StringTok{ }\ControlFlowTok{function}\NormalTok{(epigraphic_dataset)\{}
\NormalTok{  clean_text <-}\StringTok{ }\KeywordTok{gsub}\NormalTok{(}\DataTypeTok{pattern=}\NormalTok{dubious_dot_subscript[}\DecValTok{1}\NormalTok{], }\DataTypeTok{replacement=}\NormalTok{dubious_dot_subscript[}\DecValTok{2}\NormalTok{], }\DataTypeTok{x=}\NormalTok{epigraphic_dataset, }\DataTypeTok{perl=}\OtherTok{TRUE}\NormalTok{)}
\NormalTok{  clean_text <-}\StringTok{ }\KeywordTok{gsub}\NormalTok{(}\DataTypeTok{pattern=}\NormalTok{lacuna1[}\DecValTok{1}\NormalTok{], }\DataTypeTok{replacement=}\NormalTok{lacuna1[}\DecValTok{2}\NormalTok{], }\DataTypeTok{x=}\NormalTok{clean_text, }\DataTypeTok{perl=}\OtherTok{TRUE}\NormalTok{)}
\NormalTok{  clean_text <-}\StringTok{ }\KeywordTok{gsub}\NormalTok{(}\DataTypeTok{pattern=}\NormalTok{lacuna2[}\DecValTok{1}\NormalTok{], }\DataTypeTok{replacement=}\NormalTok{lacuna2[}\DecValTok{2}\NormalTok{], }\DataTypeTok{x=}\NormalTok{clean_text, }\DataTypeTok{perl=}\OtherTok{TRUE}\NormalTok{)}
\NormalTok{  clean_text <-}\StringTok{ }\KeywordTok{gsub}\NormalTok{(}\DataTypeTok{pattern=}\NormalTok{vacat[}\DecValTok{1}\NormalTok{], }\DataTypeTok{replacement=}\NormalTok{vacat[}\DecValTok{2}\NormalTok{], }\DataTypeTok{x=}\NormalTok{clean_text, }\DataTypeTok{perl=}\OtherTok{TRUE}\NormalTok{)}
\NormalTok{  clean_text <-}\StringTok{ }\KeywordTok{gsub}\NormalTok{(}\DataTypeTok{pattern=}\NormalTok{editorial_notes[}\DecValTok{1}\NormalTok{], }\DataTypeTok{replacement=}\NormalTok{editorial_notes[}\DecValTok{2}\NormalTok{], }\DataTypeTok{x=}\NormalTok{clean_text, }\DataTypeTok{perl=}\OtherTok{TRUE}\NormalTok{)}
\NormalTok{  clean_text <-}\StringTok{ }\KeywordTok{gsub}\NormalTok{(}\DataTypeTok{pattern=}\NormalTok{expanded_abbreviations_conservative[}\DecValTok{1}\NormalTok{], }\DataTypeTok{replacement=}\NormalTok{expanded_abbreviations_conservative[}\DecValTok{2}\NormalTok{], }\DataTypeTok{x=}\NormalTok{clean_text, }\DataTypeTok{perl=}\OtherTok{TRUE}\NormalTok{)}
\NormalTok{  clean_text <-}\StringTok{ }\KeywordTok{gsub}\NormalTok{(}\DataTypeTok{pattern=}\NormalTok{suppresion_superscripts_conservative[}\DecValTok{1}\NormalTok{], }\DataTypeTok{replacement=}\NormalTok{suppresion_superscripts_conservative[}\DecValTok{2}\NormalTok{], }\DataTypeTok{x=}\NormalTok{clean_text, }\DataTypeTok{perl=}\OtherTok{TRUE}\NormalTok{)}
\NormalTok{  clean_text <-}\StringTok{ }\KeywordTok{gsub}\NormalTok{(}\DataTypeTok{pattern=}\NormalTok{suppresion_conservative[}\DecValTok{1}\NormalTok{], }\DataTypeTok{replacement=}\NormalTok{suppresion_conservative[}\DecValTok{2}\NormalTok{], }\DataTypeTok{x=}\NormalTok{clean_text, }\DataTypeTok{perl=}\OtherTok{TRUE}\NormalTok{)}
\NormalTok{  clean_text <-}\StringTok{ }\KeywordTok{gsub}\NormalTok{(}\DataTypeTok{pattern=}\NormalTok{restoration_conservative[}\DecValTok{1}\NormalTok{], }\DataTypeTok{replacement=}\NormalTok{restoration_conservative[}\DecValTok{2}\NormalTok{], }\DataTypeTok{x=}\NormalTok{clean_text, }\DataTypeTok{perl=}\OtherTok{TRUE}\NormalTok{)}
\NormalTok{  clean_text <-}\StringTok{ }\KeywordTok{gsub}\NormalTok{(}\DataTypeTok{pattern=}\NormalTok{substitution_conservative[}\DecValTok{1}\NormalTok{], }\DataTypeTok{replacement=}\NormalTok{substitution_conservative[}\DecValTok{2}\NormalTok{], }\DataTypeTok{x=}\NormalTok{clean_text, }\DataTypeTok{perl=}\OtherTok{TRUE}\NormalTok{)}
\NormalTok{  clean_text <-}\StringTok{ }\KeywordTok{gsub}\NormalTok{(}\DataTypeTok{pattern=}\NormalTok{new_line[}\DecValTok{1}\NormalTok{], }\DataTypeTok{replacement=}\NormalTok{new_line[}\DecValTok{2}\NormalTok{], }\DataTypeTok{x=}\NormalTok{clean_text, }\DataTypeTok{perl=}\OtherTok{TRUE}\NormalTok{)}
\NormalTok{  clean_text <-}\StringTok{ }\KeywordTok{gsub}\NormalTok{(}\DataTypeTok{pattern=}\NormalTok{split_word_multiline[}\DecValTok{1}\NormalTok{], }\DataTypeTok{replacement=}\NormalTok{split_word_multiline[}\DecValTok{2}\NormalTok{], }\DataTypeTok{x=}\NormalTok{clean_text, }\DataTypeTok{perl=}\OtherTok{TRUE}\NormalTok{)}
\NormalTok{  clean_text <-}\StringTok{ }\KeywordTok{gsub}\NormalTok{(}\DataTypeTok{pattern=}\NormalTok{erasure_empty[}\DecValTok{1}\NormalTok{], }\DataTypeTok{replacement=}\NormalTok{erasure_empty[}\DecValTok{2}\NormalTok{], }\DataTypeTok{x=}\NormalTok{clean_text, }\DataTypeTok{perl=}\OtherTok{TRUE}\NormalTok{)}
\NormalTok{  clean_text <-}\StringTok{ }\KeywordTok{gsub}\NormalTok{(}\DataTypeTok{pattern=}\NormalTok{erasure_new_text[}\DecValTok{1}\NormalTok{], }\DataTypeTok{replacement=}\NormalTok{erasure_new_text[}\DecValTok{2}\NormalTok{], }\DataTypeTok{x=}\NormalTok{clean_text, }\DataTypeTok{perl=}\OtherTok{TRUE}\NormalTok{)}
\NormalTok{  clean_text <-}\StringTok{ }\KeywordTok{gsub}\NormalTok{(}\DataTypeTok{pattern=}\NormalTok{interpunction_symbols[}\DecValTok{1}\NormalTok{], }\DataTypeTok{replacement=}\NormalTok{interpunction_symbols[}\DecValTok{2}\NormalTok{], }\DataTypeTok{x=}\NormalTok{clean_text, }\DataTypeTok{perl=}\OtherTok{TRUE}\NormalTok{)}
\NormalTok{  clean_text <-}\StringTok{ }\KeywordTok{gsub}\NormalTok{(}\DataTypeTok{pattern=}\NormalTok{superscript_numbers[}\DecValTok{1}\NormalTok{], }\DataTypeTok{replacement=}\NormalTok{superscript_numbers[}\DecValTok{2}\NormalTok{], }\DataTypeTok{x=}\NormalTok{clean_text, }\DataTypeTok{perl=}\OtherTok{TRUE}\NormalTok{)}
\NormalTok{  clean_text <-}\StringTok{ }\KeywordTok{gsub}\NormalTok{(}\DataTypeTok{pattern=}\NormalTok{epigraphic_symbols[}\DecValTok{1}\NormalTok{], }\DataTypeTok{replacement=}\NormalTok{epigraphic_symbols[}\DecValTok{2}\NormalTok{], }\DataTypeTok{x=}\NormalTok{clean_text, }\DataTypeTok{perl=}\OtherTok{TRUE}\NormalTok{)}
\NormalTok{  clean_text <-}\StringTok{ }\KeywordTok{gsub}\NormalTok{(}\DataTypeTok{pattern=}\NormalTok{uncertainty_symbols[}\DecValTok{1}\NormalTok{], }\DataTypeTok{replacement=}\NormalTok{uncertainty_symbols[}\DecValTok{2}\NormalTok{], }\DataTypeTok{x=}\NormalTok{clean_text, }\DataTypeTok{perl=}\OtherTok{TRUE}\NormalTok{)}
\NormalTok{  clean_text <-}\StringTok{ }\KeywordTok{gsub}\NormalTok{(}\DataTypeTok{pattern=}\NormalTok{uncertainty_symbols[}\DecValTok{1}\NormalTok{], }\DataTypeTok{replacement=}\NormalTok{uncertainty_symbols[}\DecValTok{2}\NormalTok{], }\DataTypeTok{x=}\NormalTok{clean_text, }\DataTypeTok{perl=}\OtherTok{TRUE}\NormalTok{)}
\NormalTok{  clean_text <-}\StringTok{ }\KeywordTok{gsub}\NormalTok{(}\DataTypeTok{pattern=}\NormalTok{end_line[}\DecValTok{1}\NormalTok{], }\DataTypeTok{replacement=}\NormalTok{end_line[}\DecValTok{2}\NormalTok{], }\DataTypeTok{x=}\NormalTok{clean_text, }\DataTypeTok{perl=}\OtherTok{TRUE}\NormalTok{)}
\NormalTok{  clean_text <-}\StringTok{ }\KeywordTok{gsub}\NormalTok{(}\DataTypeTok{pattern=}\NormalTok{extra_blank[}\DecValTok{1}\NormalTok{], }\DataTypeTok{replacement=}\NormalTok{extra_blank[}\DecValTok{2}\NormalTok{], }\DataTypeTok{x=}\NormalTok{clean_text, }\DataTypeTok{perl=}\OtherTok{TRUE}\NormalTok{)}
\NormalTok{  clean_text <-}\StringTok{ }\KeywordTok{gsub}\NormalTok{(}\DataTypeTok{pattern=}\NormalTok{arabic_numerals[}\DecValTok{1}\NormalTok{], }\DataTypeTok{replacement=}\NormalTok{arabic_numerals[}\DecValTok{2}\NormalTok{], }\DataTypeTok{x=}\NormalTok{clean_text, }\DataTypeTok{perl=}\OtherTok{TRUE}\NormalTok{)}
\NormalTok{  clean_text <-}\StringTok{ }\KeywordTok{gsub}\NormalTok{(}\DataTypeTok{pattern=}\NormalTok{multi_whitespace[}\DecValTok{1}\NormalTok{], }\DataTypeTok{replacement=}\NormalTok{multi_whitespace[}\DecValTok{2}\NormalTok{], }\DataTypeTok{x=}\NormalTok{clean_text, }\DataTypeTok{perl=}\OtherTok{TRUE}\NormalTok{)}
\NormalTok{  clean_text <-}\StringTok{ }\KeywordTok{gsub}\NormalTok{(}\DataTypeTok{pattern=}\NormalTok{whitespace_endline[}\DecValTok{1}\NormalTok{], }\DataTypeTok{replacement=}\NormalTok{whitespace_endline[}\DecValTok{2}\NormalTok{], }\DataTypeTok{x=}\NormalTok{clean_text, }\DataTypeTok{perl=}\OtherTok{TRUE}\NormalTok{)}
      \KeywordTok{return}\NormalTok{(clean_text)}
\NormalTok{\}}
\end{Highlighting}
\end{Shaded}

\hypertarget{example-of-conservative-cleaning}{%
\paragraph{Example of conservative
cleaning:}\label{example-of-conservative-cleaning}}

\emph{Original text of an inscription IGBulg I² 15(3) before cleaning:}

{[}--- --- --- --- --- --- --- --- --- --- --- --- --- --- ---{]}

{[}--- --- ---δόντα καὶ διανομ{]}ὰ̣ς̣ τ̣ῇ̣ τ̣ε̣ κ̣ρ̣α̣-

{[}τί{]}σ̣τ̣ῃ βουλῇ καὶ ἀγορανόμοις καὶ

{[}ταῖ{]}ς ἑπτὰ φυλαῖς καὶ τοῖς ὑμνοῦσι

τοὺς Σεβαστοὺς καὶ ἀγοραίοις, ἰ-

α̣τροῖς, παιδευταῖς καὶ τοῖς παρε-

\{{[}πα{]}ρ̣ε̣\}π̣ιδη̣μήσα̣σιν \{²⁶παρεπιδημήσασιν\}²⁶ τῆ̣ς̣ Π̣ε̣ντ{[}α{]}-

{[}πόλεως βουλευταῖς --- --- --- --- ---{]}

{[}--- --- --- --- --- --- --- --- --- --- --- --- ---{]}

Output of the \texttt{cleaning\_conservative} function:

\begin{Shaded}
\begin{Highlighting}[]
\NormalTok{example_conservative <-}\StringTok{ }\KeywordTok{as.data.frame}\NormalTok{(}\KeywordTok{cleaning_conservative}\NormalTok{(dirtytext}\OperatorTok{$}\NormalTok{data))}
\NormalTok{example_conservative[}\DecValTok{30}\NormalTok{,]}
\end{Highlighting}
\end{Shaded}

\begin{verbatim}
## [1] "ὰς τῇ τε κραστῃ βουλῇ καὶ ἀγορανόμοις καὶ ς ἑπτὰ φυλαῖς καὶ τοῖς ὑμνοῦσι τοὺς Σεβαστοὺς καὶ ἀγοραίοις ἰατροῖς παιδευταῖς καὶ τοῖς παρερεπιδημήσασιν τῆς Πεντ"
\end{verbatim}

\hypertarget{interpretive-model}{%
\subsubsection{Interpretive model}\label{interpretive-model}}

\emph{Aim:} to have a clean text enriched by editorial interpretations
and reconstructions of the text (to have as rich text of an inscription
as possible).

\begin{Shaded}
\begin{Highlighting}[]
\NormalTok{cleaning_interpretive <-}\StringTok{ }\ControlFlowTok{function}\NormalTok{(epigraphic_dataset)\{}
\NormalTok{  clean_text <-}\StringTok{ }\KeywordTok{gsub}\NormalTok{(}\DataTypeTok{pattern=}\NormalTok{dubious_dot_subscript[}\DecValTok{1}\NormalTok{], }\DataTypeTok{replacement=}\NormalTok{dubious_dot_subscript[}\DecValTok{2}\NormalTok{], }\DataTypeTok{x=}\NormalTok{epigraphic_dataset, }\DataTypeTok{perl=}\OtherTok{TRUE}\NormalTok{)}
\NormalTok{  clean_text <-}\StringTok{ }\KeywordTok{gsub}\NormalTok{(}\DataTypeTok{pattern=}\NormalTok{lacuna1[}\DecValTok{1}\NormalTok{], }\DataTypeTok{replacement=}\NormalTok{lacuna1[}\DecValTok{2}\NormalTok{], }\DataTypeTok{x=}\NormalTok{clean_text, }\DataTypeTok{perl=}\OtherTok{TRUE}\NormalTok{)}
\NormalTok{  clean_text <-}\StringTok{ }\KeywordTok{gsub}\NormalTok{(}\DataTypeTok{pattern=}\NormalTok{lacuna2[}\DecValTok{1}\NormalTok{], }\DataTypeTok{replacement=}\NormalTok{lacuna2[}\DecValTok{2}\NormalTok{], }\DataTypeTok{x=}\NormalTok{clean_text, }\DataTypeTok{perl=}\OtherTok{TRUE}\NormalTok{)}
\NormalTok{  clean_text <-}\StringTok{ }\KeywordTok{gsub}\NormalTok{(}\DataTypeTok{pattern=}\NormalTok{vacat[}\DecValTok{1}\NormalTok{], }\DataTypeTok{replacement=}\NormalTok{vacat[}\DecValTok{2}\NormalTok{], }\DataTypeTok{x=}\NormalTok{clean_text, }\DataTypeTok{perl=}\OtherTok{TRUE}\NormalTok{)}
\NormalTok{  clean_text <-}\StringTok{ }\KeywordTok{gsub}\NormalTok{(}\DataTypeTok{pattern=}\NormalTok{editorial_notes[}\DecValTok{1}\NormalTok{], }\DataTypeTok{replacement=}\NormalTok{editorial_notes[}\DecValTok{2}\NormalTok{], }\DataTypeTok{x=}\NormalTok{clean_text, }\DataTypeTok{perl=}\OtherTok{TRUE}\NormalTok{)}
\NormalTok{  clean_text <-}\StringTok{ }\KeywordTok{gsub}\NormalTok{(}\DataTypeTok{pattern=}\NormalTok{editorial_comments_latin[}\DecValTok{1}\NormalTok{], }\DataTypeTok{replacement=}\NormalTok{editorial_comments_latin[}\DecValTok{2}\NormalTok{], }\DataTypeTok{x=}\NormalTok{clean_text, }\DataTypeTok{perl=}\OtherTok{TRUE}\NormalTok{)}
\NormalTok{  clean_text <-}\StringTok{ }\KeywordTok{gsub}\NormalTok{(}\DataTypeTok{pattern=}\NormalTok{expanded_abbreviations_interpretive[}\DecValTok{1}\NormalTok{], }\DataTypeTok{replacement=}\NormalTok{expanded_abbreviations_interpretive[}\DecValTok{2}\NormalTok{], }\DataTypeTok{x=}\NormalTok{clean_text, }\DataTypeTok{perl=}\OtherTok{TRUE}\NormalTok{)}
\NormalTok{ clean_text <-}\StringTok{ }\KeywordTok{gsub}\NormalTok{(}\DataTypeTok{pattern=}\NormalTok{suppresion_superscripts_interpretive[}\DecValTok{1}\NormalTok{], }\DataTypeTok{replacement=}\NormalTok{suppresion_superscripts_interpretive[}\DecValTok{2}\NormalTok{], }\DataTypeTok{x=}\NormalTok{clean_text, }\DataTypeTok{perl=}\OtherTok{TRUE}\NormalTok{)}
\NormalTok{  clean_text <-}\StringTok{ }\KeywordTok{gsub}\NormalTok{(}\DataTypeTok{pattern=}\NormalTok{suppresion_keep_interpretive[}\DecValTok{1}\NormalTok{], }\DataTypeTok{replacement=}\NormalTok{suppresion_keep_interpretive[}\DecValTok{2}\NormalTok{], }\DataTypeTok{x=}\NormalTok{clean_text, }\DataTypeTok{perl=}\OtherTok{TRUE}\NormalTok{)}
\NormalTok{  clean_text <-}\StringTok{ }\KeywordTok{gsub}\NormalTok{(}\DataTypeTok{pattern=}\NormalTok{restoration_interpretive[}\DecValTok{1}\NormalTok{], }\DataTypeTok{replacement=}\NormalTok{restoration_interpretive[}\DecValTok{2}\NormalTok{], }\DataTypeTok{x=}\NormalTok{clean_text, }\DataTypeTok{perl=}\OtherTok{TRUE}\NormalTok{)}
\NormalTok{  clean_text <-}\StringTok{ }\KeywordTok{gsub}\NormalTok{(}\DataTypeTok{pattern=}\NormalTok{substitution_interpretive[}\DecValTok{1}\NormalTok{], }\DataTypeTok{replacement=}\NormalTok{substitution_interpretive[}\DecValTok{2}\NormalTok{], }\DataTypeTok{x=}\NormalTok{clean_text, }\DataTypeTok{perl=}\OtherTok{TRUE}\NormalTok{)}
\NormalTok{  clean_text <-}\StringTok{ }\KeywordTok{gsub}\NormalTok{(}\DataTypeTok{pattern=}\NormalTok{new_line[}\DecValTok{1}\NormalTok{], }\DataTypeTok{replacement=}\NormalTok{new_line[}\DecValTok{2}\NormalTok{], }\DataTypeTok{x=}\NormalTok{clean_text, }\DataTypeTok{perl=}\OtherTok{TRUE}\NormalTok{)}
\NormalTok{  clean_text <-}\StringTok{ }\KeywordTok{gsub}\NormalTok{(}\DataTypeTok{pattern=}\NormalTok{split_word_multiline[}\DecValTok{1}\NormalTok{], }\DataTypeTok{replacement=}\NormalTok{split_word_multiline[}\DecValTok{2}\NormalTok{], }\DataTypeTok{x=}\NormalTok{clean_text, }\DataTypeTok{perl=}\OtherTok{TRUE}\NormalTok{)}
\NormalTok{  clean_text <-}\StringTok{ }\KeywordTok{gsub}\NormalTok{(}\DataTypeTok{pattern=}\NormalTok{erasure_empty[}\DecValTok{1}\NormalTok{], }\DataTypeTok{replacement=}\NormalTok{erasure_empty[}\DecValTok{2}\NormalTok{], }\DataTypeTok{x=}\NormalTok{clean_text, }\DataTypeTok{perl=}\OtherTok{TRUE}\NormalTok{)}
\NormalTok{  clean_text <-}\StringTok{ }\KeywordTok{gsub}\NormalTok{(}\DataTypeTok{pattern=}\NormalTok{erasure_new_text[}\DecValTok{1}\NormalTok{], }\DataTypeTok{replacement=}\NormalTok{erasure_new_text[}\DecValTok{2}\NormalTok{], }\DataTypeTok{x=}\NormalTok{clean_text, }\DataTypeTok{perl=}\OtherTok{TRUE}\NormalTok{)}
\NormalTok{  clean_text <-}\StringTok{ }\KeywordTok{gsub}\NormalTok{(}\DataTypeTok{pattern=}\NormalTok{interpunction_symbols[}\DecValTok{1}\NormalTok{], }\DataTypeTok{replacement=}\NormalTok{interpunction_symbols[}\DecValTok{2}\NormalTok{], }\DataTypeTok{x=}\NormalTok{clean_text, }\DataTypeTok{perl=}\OtherTok{TRUE}\NormalTok{)}
\NormalTok{  clean_text <-}\StringTok{ }\KeywordTok{gsub}\NormalTok{(}\DataTypeTok{pattern=}\NormalTok{superscript_numbers[}\DecValTok{1}\NormalTok{], }\DataTypeTok{replacement=}\NormalTok{superscript_numbers[}\DecValTok{2}\NormalTok{], }\DataTypeTok{x=}\NormalTok{clean_text, }\DataTypeTok{perl=}\OtherTok{TRUE}\NormalTok{)}
\NormalTok{  clean_text <-}\StringTok{ }\KeywordTok{gsub}\NormalTok{(}\DataTypeTok{pattern=}\NormalTok{epigraphic_symbols[}\DecValTok{1}\NormalTok{], }\DataTypeTok{replacement=}\NormalTok{epigraphic_symbols[}\DecValTok{2}\NormalTok{], }\DataTypeTok{x=}\NormalTok{clean_text, }\DataTypeTok{perl=}\OtherTok{TRUE}\NormalTok{)}
\NormalTok{  clean_text <-}\StringTok{ }\KeywordTok{gsub}\NormalTok{(}\DataTypeTok{pattern=}\NormalTok{uncertainty_symbols[}\DecValTok{1}\NormalTok{], }\DataTypeTok{replacement=}\NormalTok{uncertainty_symbols[}\DecValTok{2}\NormalTok{], }\DataTypeTok{x=}\NormalTok{clean_text, }\DataTypeTok{perl=}\OtherTok{TRUE}\NormalTok{)}
\NormalTok{  clean_text <-}\StringTok{ }\KeywordTok{gsub}\NormalTok{(}\DataTypeTok{pattern=}\NormalTok{end_line[}\DecValTok{1}\NormalTok{], }\DataTypeTok{replacement=}\NormalTok{end_line[}\DecValTok{2}\NormalTok{], }\DataTypeTok{x=}\NormalTok{clean_text, }\DataTypeTok{perl=}\OtherTok{TRUE}\NormalTok{)}
\NormalTok{  clean_text <-}\StringTok{ }\KeywordTok{gsub}\NormalTok{(}\DataTypeTok{pattern=}\NormalTok{extra_blank[}\DecValTok{1}\NormalTok{], }\DataTypeTok{replacement=}\NormalTok{extra_blank[}\DecValTok{2}\NormalTok{], }\DataTypeTok{x=}\NormalTok{clean_text, }\DataTypeTok{perl=}\OtherTok{TRUE}\NormalTok{)}
\NormalTok{  clean_text <-}\StringTok{ }\KeywordTok{gsub}\NormalTok{(}\DataTypeTok{pattern=}\NormalTok{arabic_numerals[}\DecValTok{1}\NormalTok{], }\DataTypeTok{replacement=}\NormalTok{arabic_numerals[}\DecValTok{2}\NormalTok{], }\DataTypeTok{x=}\NormalTok{clean_text, }\DataTypeTok{perl=}\OtherTok{TRUE}\NormalTok{)}
\NormalTok{  clean_text <-}\StringTok{ }\KeywordTok{gsub}\NormalTok{(}\DataTypeTok{pattern=}\NormalTok{multi_whitespace[}\DecValTok{1}\NormalTok{], }\DataTypeTok{replacement=}\NormalTok{multi_whitespace[}\DecValTok{2}\NormalTok{], }\DataTypeTok{x=}\NormalTok{clean_text, }\DataTypeTok{perl=}\OtherTok{TRUE}\NormalTok{)}
\NormalTok{  clean_text <-}\StringTok{ }\KeywordTok{gsub}\NormalTok{(}\DataTypeTok{pattern=}\NormalTok{whitespace_endline[}\DecValTok{1}\NormalTok{], }\DataTypeTok{replacement=}\NormalTok{whitespace_endline[}\DecValTok{2}\NormalTok{], }\DataTypeTok{x=}\NormalTok{clean_text, }\DataTypeTok{perl=}\OtherTok{TRUE}\NormalTok{)}
      \KeywordTok{return}\NormalTok{(clean_text)}
\NormalTok{\}}
\end{Highlighting}
\end{Shaded}

\hypertarget{example-of-interpretive-cleaning}{%
\paragraph{Example of interpretive
cleaning:}\label{example-of-interpretive-cleaning}}

\emph{Original text of an inscription IGBulg I² 15(3) before cleaning:}

{[}--- --- --- --- --- --- --- --- --- --- --- --- --- --- ---{]}

{[}--- --- ---δόντα καὶ διανομ{]}ὰ̣ς̣ τ̣ῇ̣ τ̣ε̣ κ̣ρ̣α̣-

{[}τί{]}σ̣τ̣ῃ βουλῇ καὶ ἀγορανόμοις καὶ

{[}ταῖ{]}ς ἑπτὰ φυλαῖς καὶ τοῖς ὑμνοῦσι

τοὺς Σεβαστοὺς καὶ ἀγοραίοις, ἰ-

α̣τροῖς, παιδευταῖς καὶ τοῖς παρε-

\{{[}πα{]}ρ̣ε̣\}π̣ιδη̣μήσα̣σιν \{²⁶παρεπιδημήσασιν\}²⁶ τῆ̣ς̣ Π̣ε̣ντ{[}α{]}-

{[}πόλεως βουλευταῖς --- --- --- --- ---{]}

{[}--- --- --- --- --- --- --- --- --- --- --- --- ---{]}

Output of the \texttt{cleanining\_interpretive} function:

\begin{Shaded}
\begin{Highlighting}[]
\NormalTok{example_interpretive <-}\StringTok{ }\KeywordTok{as.data.frame}\NormalTok{(}\KeywordTok{cleaning_interpretive}\NormalTok{(dirtytext}\OperatorTok{$}\NormalTok{data))}
\NormalTok{example_interpretive[}\DecValTok{30}\NormalTok{,]}
\end{Highlighting}
\end{Shaded}

\begin{verbatim}
## [1] "δόντα καὶ διανομὰς τῇ τε κρατίστῃ βουλῇ καὶ ἀγορανόμοις καὶ ταῖς ἑπτὰ φυλαῖς καὶ τοῖς ὑμνοῦσι τοὺς Σεβαστοὺς καὶ ἀγοραίοις ἰατροῖς παιδευταῖς καὶ τοῖς παρεπιδημήσασιν τῆς Πενταπόλεως βουλευταῖς"
\end{verbatim}

\hypertarget{saving-both-versions-of-the-cleaned-text-in-a-csv}{%
\subsubsection{Saving both versions of the cleaned text in a
CSV}\label{saving-both-versions-of-the-cleaned-text-in-a-csv}}

Save the output of \texttt{cleaning\_conservative} and
\texttt{cleaning\_interpretive} function together with the original
contents of the dataset. Create a new directory \texttt{outputs} in the
root folder if it does not exist.

\begin{Shaded}
\begin{Highlighting}[]
\NormalTok{clean_text <-}\StringTok{ }\NormalTok{text }\OperatorTok
\StringTok{  }\KeywordTok{mutate}\NormalTok{(}\DataTypeTok{clean_text_conservative =} \KeywordTok{cleaning_conservative}\NormalTok{(text}\OperatorTok{$}\NormalTok{data)) }\OperatorTok
\StringTok{  }\KeywordTok{mutate}\NormalTok{(}\DataTypeTok{clean_text_interpretive =} \KeywordTok{cleaning_interpretive}\NormalTok{(text}\OperatorTok{$}\NormalTok{data))}
\end{Highlighting}
\end{Shaded}

\begin{Shaded}
\begin{Highlighting}[]
\CommentTok{# dir.create("../../outputs")}
\KeywordTok{write_csv}\NormalTok{(clean_text, }\DataTypeTok{path =} \StringTok{"../../outputs/PHI_IGBulg-I_clean_text.csv"}\NormalTok{)}
\end{Highlighting}
\end{Shaded}

\hypertarget{edh-inscriptions-dataset}{%
\subsection{EDH Inscriptions dataset}\label{edh-inscriptions-dataset}}

Source: \url{https://edh-www.adw.uni-heidelberg.de/}

\hypertarget{loading-data-1}{%
\subsubsection{Loading data}\label{loading-data-1}}

First, we need to install several more packages and load the libraries
in order to connect to Sciencedata.dk and access the dataset.

\begin{enumerate}
\def\labelenumi{\arabic{enumi}.}
\tightlist
\item
  Input your sciencedata.dk username - type directly into the RStudio
  console
\end{enumerate}

\begin{verbatim}
## your sciencedata username:
\end{verbatim}

\begin{enumerate}
\def\labelenumi{\arabic{enumi}.}
\setcounter{enumi}{1}
\tightlist
\item
  Make the request (you will be asked for password in a new pop-up
  window)
\end{enumerate}

\begin{verbatim}
## Please enter password in TK window (Alt+Tab)
\end{verbatim}

Sample data for testing (5000 inscriptions only)

\begin{verbatim}
## Please enter password in TK window (Alt+Tab)
\end{verbatim}

\begin{enumerate}
\def\labelenumi{\arabic{enumi}.}
\setcounter{enumi}{2}
\tightlist
\item
  Make a list from the request and display the first six records (head)
\end{enumerate}

\begin{Shaded}
\begin{Highlighting}[]
\NormalTok{list_json <-}\StringTok{ }\KeywordTok{fromJSON}\NormalTok{(resp)}
\end{Highlighting}
\end{Shaded}

\begin{verbatim}
## Error in fromJSON(resp): unexpected character '<'
\end{verbatim}

\begin{Shaded}
\begin{Highlighting}[]
\NormalTok{EDH_tibble =}\StringTok{ }\KeywordTok{as_tibble}\NormalTok{(list_json)}
\end{Highlighting}
\end{Shaded}

\begin{verbatim}
## Error in as_tibble(list_json): object 'list_json' not found
\end{verbatim}

\begin{Shaded}
\begin{Highlighting}[]
\KeywordTok{head}\NormalTok{(EDH_tibble)}
\end{Highlighting}
\end{Shaded}

\begin{verbatim}
## Error in head(EDH_tibble): object 'EDH_tibble' not found
\end{verbatim}

\hypertarget{conservative-model-1}{%
\subsubsection{Conservative model}\label{conservative-model-1}}

\emph{Aim:} to have a clean text that is as close to the original
inscription as preserved on the medium - in case of the EDH dataset
column \texttt{diplomatic\_text} should be similar to the output of the
\texttt{conservative\_cleaning} model.

Since the dataset is mostly in Latin, I did not use the following
cleaning scripts: \texttt{vacat}, \texttt{editorial\_notes},
\texttt{editorial\_comments\_latin} since they would eliminate some
parts of the text that should not be eliminated. I am not using the
\texttt{suppresion\_superscripts\_conservative} script beacuse the
structure of the EDH dataset does not contain curly braces followed by
superscript numbers. The script \texttt{unclosed\_brackets} has been
added since EDH dataset contains a lot of unclosed brackets of all
kinds. Script \texttt{substitution\_edh\_conservative} was added to
clean additional substitution features of the EDH dataset.

\begin{Shaded}
\begin{Highlighting}[]
\NormalTok{cleaning_conservative_edh <-}\StringTok{ }\ControlFlowTok{function}\NormalTok{(epigraphic_dataset)\{}
\NormalTok{  clean_text <-}\StringTok{ }\KeywordTok{gsub}\NormalTok{(}\DataTypeTok{pattern=}\NormalTok{dubious_dot_subscript[}\DecValTok{1}\NormalTok{], }\DataTypeTok{replacement=}\NormalTok{dubious_dot_subscript[}\DecValTok{2}\NormalTok{], }\DataTypeTok{x=}\NormalTok{epigraphic_dataset, }\DataTypeTok{perl=}\OtherTok{TRUE}\NormalTok{)}
\NormalTok{  clean_text <-}\StringTok{ }\KeywordTok{gsub}\NormalTok{(}\DataTypeTok{pattern=}\NormalTok{lacuna1[}\DecValTok{1}\NormalTok{], }\DataTypeTok{replacement=}\NormalTok{lacuna1[}\DecValTok{2}\NormalTok{], }\DataTypeTok{x=}\NormalTok{clean_text, }\DataTypeTok{perl=}\OtherTok{TRUE}\NormalTok{)}
\NormalTok{  clean_text <-}\StringTok{ }\KeywordTok{gsub}\NormalTok{(}\DataTypeTok{pattern=}\NormalTok{lacuna2[}\DecValTok{1}\NormalTok{], }\DataTypeTok{replacement=}\NormalTok{lacuna2[}\DecValTok{2}\NormalTok{], }\DataTypeTok{x=}\NormalTok{clean_text, }\DataTypeTok{perl=}\OtherTok{TRUE}\NormalTok{)}
\NormalTok{  clean_text <-}\StringTok{ }\KeywordTok{gsub}\NormalTok{(}\DataTypeTok{pattern=}\NormalTok{expanded_abbreviations_conservative[}\DecValTok{1}\NormalTok{], }\DataTypeTok{replacement=}\NormalTok{expanded_abbreviations_conservative[}\DecValTok{2}\NormalTok{], }\DataTypeTok{x=}\NormalTok{clean_text, }\DataTypeTok{perl=}\OtherTok{TRUE}\NormalTok{)}
\NormalTok{  clean_text <-}\StringTok{ }\KeywordTok{gsub}\NormalTok{(}\DataTypeTok{pattern=}\NormalTok{suppresion_conservative[}\DecValTok{1}\NormalTok{], }\DataTypeTok{replacement=}\NormalTok{suppresion_conservative[}\DecValTok{2}\NormalTok{], }\DataTypeTok{x=}\NormalTok{clean_text, }\DataTypeTok{perl=}\OtherTok{TRUE}\NormalTok{)}
\NormalTok{  clean_text <-}\StringTok{ }\KeywordTok{gsub}\NormalTok{(}\DataTypeTok{pattern=}\NormalTok{restoration_conservative[}\DecValTok{1}\NormalTok{], }\DataTypeTok{replacement=}\NormalTok{restoration_conservative[}\DecValTok{2}\NormalTok{], }\DataTypeTok{x=}\NormalTok{clean_text, }\DataTypeTok{perl=}\OtherTok{TRUE}\NormalTok{)}
\NormalTok{  clean_text <-}\StringTok{ }\KeywordTok{gsub}\NormalTok{(}\DataTypeTok{pattern=}\NormalTok{substitution_edh_conservative[}\DecValTok{1}\NormalTok{], }\DataTypeTok{replacement=}\NormalTok{substitution_edh_conservative[}\DecValTok{2}\NormalTok{], }\DataTypeTok{x=}\NormalTok{clean_text, }\DataTypeTok{perl=}\OtherTok{TRUE}\NormalTok{)}
\NormalTok{  clean_text <-}\StringTok{ }\KeywordTok{gsub}\NormalTok{(}\DataTypeTok{pattern=}\NormalTok{substitution_conservative[}\DecValTok{1}\NormalTok{], }\DataTypeTok{replacement=}\NormalTok{substitution_conservative[}\DecValTok{2}\NormalTok{], }\DataTypeTok{x=}\NormalTok{clean_text, }\DataTypeTok{perl=}\OtherTok{TRUE}\NormalTok{)}
\NormalTok{  clean_text <-}\StringTok{ }\KeywordTok{gsub}\NormalTok{(}\DataTypeTok{pattern=}\NormalTok{new_line[}\DecValTok{1}\NormalTok{], }\DataTypeTok{replacement=}\NormalTok{new_line[}\DecValTok{2}\NormalTok{], }\DataTypeTok{x=}\NormalTok{clean_text, }\DataTypeTok{perl=}\OtherTok{TRUE}\NormalTok{)}
\NormalTok{  clean_text <-}\StringTok{ }\KeywordTok{gsub}\NormalTok{(}\DataTypeTok{pattern=}\NormalTok{split_word_multiline[}\DecValTok{1}\NormalTok{], }\DataTypeTok{replacement=}\NormalTok{split_word_multiline[}\DecValTok{2}\NormalTok{], }\DataTypeTok{x=}\NormalTok{clean_text, }\DataTypeTok{perl=}\OtherTok{TRUE}\NormalTok{)}
\NormalTok{  clean_text <-}\StringTok{ }\KeywordTok{gsub}\NormalTok{(}\DataTypeTok{pattern=}\NormalTok{erasure_empty[}\DecValTok{1}\NormalTok{], }\DataTypeTok{replacement=}\NormalTok{erasure_empty[}\DecValTok{2}\NormalTok{], }\DataTypeTok{x=}\NormalTok{clean_text, }\DataTypeTok{perl=}\OtherTok{TRUE}\NormalTok{)}
\NormalTok{  clean_text <-}\StringTok{ }\KeywordTok{gsub}\NormalTok{(}\DataTypeTok{pattern=}\NormalTok{erasure_new_text[}\DecValTok{1}\NormalTok{], }\DataTypeTok{replacement=}\NormalTok{erasure_new_text[}\DecValTok{2}\NormalTok{], }\DataTypeTok{x=}\NormalTok{clean_text, }\DataTypeTok{perl=}\OtherTok{TRUE}\NormalTok{)}
\NormalTok{  clean_text <-}\StringTok{ }\KeywordTok{gsub}\NormalTok{(}\DataTypeTok{pattern=}\NormalTok{interpunction_symbols[}\DecValTok{1}\NormalTok{], }\DataTypeTok{replacement=}\NormalTok{interpunction_symbols[}\DecValTok{2}\NormalTok{], }\DataTypeTok{x=}\NormalTok{clean_text, }\DataTypeTok{perl=}\OtherTok{TRUE}\NormalTok{)}
\NormalTok{  clean_text <-}\StringTok{ }\KeywordTok{gsub}\NormalTok{(}\DataTypeTok{pattern=}\NormalTok{superscript_numbers[}\DecValTok{1}\NormalTok{], }\DataTypeTok{replacement=}\NormalTok{superscript_numbers[}\DecValTok{2}\NormalTok{], }\DataTypeTok{x=}\NormalTok{clean_text, }\DataTypeTok{perl=}\OtherTok{TRUE}\NormalTok{)}
\NormalTok{  clean_text <-}\StringTok{ }\KeywordTok{gsub}\NormalTok{(}\DataTypeTok{pattern=}\NormalTok{epigraphic_symbols[}\DecValTok{1}\NormalTok{], }\DataTypeTok{replacement=}\NormalTok{epigraphic_symbols[}\DecValTok{2}\NormalTok{], }\DataTypeTok{x=}\NormalTok{clean_text, }\DataTypeTok{perl=}\OtherTok{TRUE}\NormalTok{)}
\NormalTok{  clean_text <-}\StringTok{ }\KeywordTok{gsub}\NormalTok{(}\DataTypeTok{pattern=}\NormalTok{uncertainty_symbols[}\DecValTok{1}\NormalTok{], }\DataTypeTok{replacement=}\NormalTok{uncertainty_symbols[}\DecValTok{2}\NormalTok{], }\DataTypeTok{x=}\NormalTok{clean_text, }\DataTypeTok{perl=}\OtherTok{TRUE}\NormalTok{)}
\NormalTok{  clean_text <-}\StringTok{ }\KeywordTok{gsub}\NormalTok{(}\DataTypeTok{pattern=}\NormalTok{uncertainty_symbols[}\DecValTok{1}\NormalTok{], }\DataTypeTok{replacement=}\NormalTok{uncertainty_symbols[}\DecValTok{2}\NormalTok{], }\DataTypeTok{x=}\NormalTok{clean_text, }\DataTypeTok{perl=}\OtherTok{TRUE}\NormalTok{)}
\NormalTok{  clean_text <-}\StringTok{ }\KeywordTok{gsub}\NormalTok{(}\DataTypeTok{pattern=}\NormalTok{end_line[}\DecValTok{1}\NormalTok{], }\DataTypeTok{replacement=}\NormalTok{end_line[}\DecValTok{2}\NormalTok{], }\DataTypeTok{x=}\NormalTok{clean_text, }\DataTypeTok{perl=}\OtherTok{TRUE}\NormalTok{)}
\NormalTok{  clean_text <-}\StringTok{ }\KeywordTok{gsub}\NormalTok{(}\DataTypeTok{pattern=}\NormalTok{extra_blank[}\DecValTok{1}\NormalTok{], }\DataTypeTok{replacement=}\NormalTok{extra_blank[}\DecValTok{2}\NormalTok{], }\DataTypeTok{x=}\NormalTok{clean_text, }\DataTypeTok{perl=}\OtherTok{TRUE}\NormalTok{)}
\NormalTok{  clean_text <-}\StringTok{ }\KeywordTok{gsub}\NormalTok{(}\DataTypeTok{pattern=}\NormalTok{arabic_numerals[}\DecValTok{1}\NormalTok{], }\DataTypeTok{replacement=}\NormalTok{arabic_numerals[}\DecValTok{2}\NormalTok{], }\DataTypeTok{x=}\NormalTok{clean_text, }\DataTypeTok{perl=}\OtherTok{TRUE}\NormalTok{)}
\NormalTok{  clean_text <-}\StringTok{ }\KeywordTok{gsub}\NormalTok{(}\DataTypeTok{pattern=}\NormalTok{unclosed_brackets[}\DecValTok{1}\NormalTok{], }\DataTypeTok{replacement=}\NormalTok{unclosed_brackets[}\DecValTok{2}\NormalTok{], }\DataTypeTok{x=}\NormalTok{clean_text, }\DataTypeTok{perl=}\OtherTok{TRUE}\NormalTok{)}
\NormalTok{  clean_text <-}\StringTok{ }\KeywordTok{gsub}\NormalTok{(}\DataTypeTok{pattern=}\NormalTok{multi_whitespace[}\DecValTok{1}\NormalTok{], }\DataTypeTok{replacement=}\NormalTok{multi_whitespace[}\DecValTok{2}\NormalTok{], }\DataTypeTok{x=}\NormalTok{clean_text, }\DataTypeTok{perl=}\OtherTok{TRUE}\NormalTok{)}
\NormalTok{  clean_text <-}\StringTok{ }\KeywordTok{gsub}\NormalTok{(}\DataTypeTok{pattern=}\NormalTok{whitespace_endline[}\DecValTok{1}\NormalTok{], }\DataTypeTok{replacement=}\NormalTok{whitespace_endline[}\DecValTok{2}\NormalTok{], }\DataTypeTok{x=}\NormalTok{clean_text, }\DataTypeTok{perl=}\OtherTok{TRUE}\NormalTok{)}
  \KeywordTok{return}\NormalTok{(clean_text)}
\NormalTok{\}}
\end{Highlighting}
\end{Shaded}

\hypertarget{example-of-conservative-cleaning-1}{%
\paragraph{Example of conservative
cleaning:}\label{example-of-conservative-cleaning-1}}

\texttt{Transcription} column of the first five inscriptions before
cleaning:

\begin{Shaded}
\begin{Highlighting}[]
\KeywordTok{print}\NormalTok{(EDH_tibble}\OperatorTok{$}\NormalTok{transcription[}\DecValTok{1}\OperatorTok{:}\DecValTok{5}\NormalTok{])}
\end{Highlighting}
\end{Shaded}

\begin{verbatim}
## Error in print(EDH_tibble$transcription[1:5]): object 'EDH_tibble' not found
\end{verbatim}

\texttt{Diplomatic\_text} column of the first five inscriptions (for
comparison with the cleaning output):

\begin{Shaded}
\begin{Highlighting}[]
\KeywordTok{print}\NormalTok{(EDH_tibble}\OperatorTok{$}\NormalTok{diplomatic_text[}\DecValTok{1}\OperatorTok{:}\DecValTok{5}\NormalTok{])}
\end{Highlighting}
\end{Shaded}

\begin{verbatim}
## Error in print(EDH_tibble$diplomatic_text[1:5]): object 'EDH_tibble' not found
\end{verbatim}

Output of the \texttt{cleaning\_conservative\_edh} function:

\begin{Shaded}
\begin{Highlighting}[]
\NormalTok{example_edh <-}\StringTok{ }\KeywordTok{as.data.frame}\NormalTok{(}\KeywordTok{cleaning_conservative_edh}\NormalTok{(EDH_tibble}\OperatorTok{$}\NormalTok{transcription))}
\end{Highlighting}
\end{Shaded}

\begin{verbatim}
## Error in gsub(pattern = dubious_dot_subscript[1], replacement = dubious_dot_subscript[2], : object 'EDH_tibble' not found
\end{verbatim}

\begin{Shaded}
\begin{Highlighting}[]
\NormalTok{example_edh[}\DecValTok{1}\OperatorTok{:}\DecValTok{5}\NormalTok{,]}
\end{Highlighting}
\end{Shaded}

\begin{verbatim}
## Error in eval(expr, envir, enclos): object 'example_edh' not found
\end{verbatim}

\hypertarget{interpretive-model-1}{%
\subsubsection{Interpretive model}\label{interpretive-model-1}}

\emph{Aim:} to have a clean text enriched by editorial interpretations
and reconstructions of the text (to have as rich text of an inscription
as possible).

Since the dataset is mostly in Latin, I did not use the following
cleaning scripts: \texttt{vacat}, \texttt{editorial\_notes},
\texttt{editorial\_comments\_latin} since they would eliminate some
parts of the text that should not be eliminated. I am not using the
\texttt{suppresion\_superscripts\_interpretive} script beacuse the
structure of the EDH dataset does not contain curly braces followed by
superscript numbers. The script \texttt{unclosed\_brackets} has been
added since EDH dataset contains a lot of unclosed brackets of all
kinds. Script \texttt{substitution\_edh\_interpretive} was added to
clean additional substitution features of the EDH dataset.

EDH has provided their own version of clean text in the column
\texttt{text\_cleaned} but did not provide any cleaning script or steps
leading to the current state of \texttt{text\_cleaned}. As a second step
I will compare the output of the \texttt{interpretive\_cleaning} model
with the \texttt{text\_cleaned} version to see who has produced better
text for text mining.

\begin{Shaded}
\begin{Highlighting}[]
\NormalTok{cleaning_interpretive_edh <-}\StringTok{ }\ControlFlowTok{function}\NormalTok{(epigraphic_dataset)\{}
\NormalTok{   clean_text <-}\StringTok{ }\KeywordTok{gsub}\NormalTok{(}\DataTypeTok{pattern=}\NormalTok{dubious_dot_subscript[}\DecValTok{1}\NormalTok{], }\DataTypeTok{replacement=}\NormalTok{dubious_dot_subscript[}\DecValTok{2}\NormalTok{], }\DataTypeTok{x=}\NormalTok{epigraphic_dataset, }\DataTypeTok{perl=}\OtherTok{TRUE}\NormalTok{)}
\NormalTok{   clean_text <-}\StringTok{ }\KeywordTok{gsub}\NormalTok{(}\DataTypeTok{pattern=}\NormalTok{lacuna1[}\DecValTok{1}\NormalTok{], }\DataTypeTok{replacement=}\NormalTok{lacuna1[}\DecValTok{2}\NormalTok{], }\DataTypeTok{x=}\NormalTok{clean_text, }\DataTypeTok{perl=}\OtherTok{TRUE}\NormalTok{)}
\NormalTok{   clean_text <-}\StringTok{ }\KeywordTok{gsub}\NormalTok{(}\DataTypeTok{pattern=}\NormalTok{lacuna2[}\DecValTok{1}\NormalTok{], }\DataTypeTok{replacement=}\NormalTok{lacuna2[}\DecValTok{2}\NormalTok{], }\DataTypeTok{x=}\NormalTok{clean_text, }\DataTypeTok{perl=}\OtherTok{TRUE}\NormalTok{)}
\NormalTok{   clean_text <-}\StringTok{ }\KeywordTok{gsub}\NormalTok{(}\DataTypeTok{pattern=}\NormalTok{expanded_abbreviations_interpretive[}\DecValTok{1}\NormalTok{], }\DataTypeTok{replacement=}\NormalTok{expanded_abbreviations_interpretive[}\DecValTok{2}\NormalTok{], }\DataTypeTok{x=}\NormalTok{clean_text, }\DataTypeTok{perl=}\OtherTok{TRUE}\NormalTok{)}
\NormalTok{   clean_text <-}\StringTok{ }\KeywordTok{gsub}\NormalTok{(}\DataTypeTok{pattern=}\NormalTok{suppresion_keep_interpretive[}\DecValTok{1}\NormalTok{], }\DataTypeTok{replacement=}\NormalTok{suppresion_keep_interpretive[}\DecValTok{2}\NormalTok{], }\DataTypeTok{x=}\NormalTok{clean_text, }\DataTypeTok{perl=}\OtherTok{TRUE}\NormalTok{)}
\NormalTok{   clean_text <-}\StringTok{ }\KeywordTok{gsub}\NormalTok{(}\DataTypeTok{pattern=}\NormalTok{restoration_interpretive[}\DecValTok{1}\NormalTok{], }\DataTypeTok{replacement=}\NormalTok{restoration_interpretive[}\DecValTok{2}\NormalTok{], }\DataTypeTok{x=}\NormalTok{clean_text, }\DataTypeTok{perl=}\OtherTok{TRUE}\NormalTok{)}
\NormalTok{   clean_text <-}\StringTok{ }\KeywordTok{gsub}\NormalTok{(}\DataTypeTok{pattern=}\NormalTok{substitution_edh_interpretive[}\DecValTok{1}\NormalTok{], }\DataTypeTok{replacement=}\NormalTok{substitution_edh_interpretive[}\DecValTok{2}\NormalTok{], }\DataTypeTok{x=}\NormalTok{clean_text, }\DataTypeTok{perl=}\OtherTok{TRUE}\NormalTok{)}
\NormalTok{   clean_text <-}\StringTok{ }\KeywordTok{gsub}\NormalTok{(}\DataTypeTok{pattern=}\NormalTok{substitution_interpretive[}\DecValTok{1}\NormalTok{], }\DataTypeTok{replacement=}\NormalTok{substitution_interpretive[}\DecValTok{2}\NormalTok{], }\DataTypeTok{x=}\NormalTok{clean_text, }\DataTypeTok{perl=}\OtherTok{TRUE}\NormalTok{)}
\NormalTok{   clean_text <-}\StringTok{ }\KeywordTok{gsub}\NormalTok{(}\DataTypeTok{pattern=}\NormalTok{new_line[}\DecValTok{1}\NormalTok{], }\DataTypeTok{replacement=}\NormalTok{new_line[}\DecValTok{2}\NormalTok{], }\DataTypeTok{x=}\NormalTok{clean_text, }\DataTypeTok{perl=}\OtherTok{TRUE}\NormalTok{)}
\NormalTok{   clean_text <-}\StringTok{ }\KeywordTok{gsub}\NormalTok{(}\DataTypeTok{pattern=}\NormalTok{split_word_multiline[}\DecValTok{1}\NormalTok{], }\DataTypeTok{replacement=}\NormalTok{split_word_multiline[}\DecValTok{2}\NormalTok{], }\DataTypeTok{x=}\NormalTok{clean_text, }\DataTypeTok{perl=}\OtherTok{TRUE}\NormalTok{)}
\NormalTok{   clean_text <-}\StringTok{ }\KeywordTok{gsub}\NormalTok{(}\DataTypeTok{pattern=}\NormalTok{erasure_empty[}\DecValTok{1}\NormalTok{], }\DataTypeTok{replacement=}\NormalTok{erasure_empty[}\DecValTok{2}\NormalTok{], }\DataTypeTok{x=}\NormalTok{clean_text, }\DataTypeTok{perl=}\OtherTok{TRUE}\NormalTok{)}
\NormalTok{   clean_text <-}\StringTok{ }\KeywordTok{gsub}\NormalTok{(}\DataTypeTok{pattern=}\NormalTok{erasure_new_text[}\DecValTok{1}\NormalTok{], }\DataTypeTok{replacement=}\NormalTok{erasure_new_text[}\DecValTok{2}\NormalTok{], }\DataTypeTok{x=}\NormalTok{clean_text, }\DataTypeTok{perl=}\OtherTok{TRUE}\NormalTok{)}
\NormalTok{   clean_text <-}\StringTok{ }\KeywordTok{gsub}\NormalTok{(}\DataTypeTok{pattern=}\NormalTok{interpunction_symbols[}\DecValTok{1}\NormalTok{], }\DataTypeTok{replacement=}\NormalTok{interpunction_symbols[}\DecValTok{2}\NormalTok{], }\DataTypeTok{x=}\NormalTok{clean_text, }\DataTypeTok{perl=}\OtherTok{TRUE}\NormalTok{)}
\NormalTok{   clean_text <-}\StringTok{ }\KeywordTok{gsub}\NormalTok{(}\DataTypeTok{pattern=}\NormalTok{superscript_numbers[}\DecValTok{1}\NormalTok{], }\DataTypeTok{replacement=}\NormalTok{superscript_numbers[}\DecValTok{2}\NormalTok{], }\DataTypeTok{x=}\NormalTok{clean_text, }\DataTypeTok{perl=}\OtherTok{TRUE}\NormalTok{)}
\NormalTok{   clean_text <-}\StringTok{ }\KeywordTok{gsub}\NormalTok{(}\DataTypeTok{pattern=}\NormalTok{epigraphic_symbols[}\DecValTok{1}\NormalTok{], }\DataTypeTok{replacement=}\NormalTok{epigraphic_symbols[}\DecValTok{2}\NormalTok{], }\DataTypeTok{x=}\NormalTok{clean_text, }\DataTypeTok{perl=}\OtherTok{TRUE}\NormalTok{)}
\NormalTok{   clean_text <-}\StringTok{ }\KeywordTok{gsub}\NormalTok{(}\DataTypeTok{pattern=}\NormalTok{uncertainty_symbols[}\DecValTok{1}\NormalTok{], }\DataTypeTok{replacement=}\NormalTok{uncertainty_symbols[}\DecValTok{2}\NormalTok{], }\DataTypeTok{x=}\NormalTok{clean_text, }\DataTypeTok{perl=}\OtherTok{TRUE}\NormalTok{)}
\NormalTok{   clean_text <-}\StringTok{ }\KeywordTok{gsub}\NormalTok{(}\DataTypeTok{pattern=}\NormalTok{end_line[}\DecValTok{1}\NormalTok{], }\DataTypeTok{replacement=}\NormalTok{end_line[}\DecValTok{2}\NormalTok{], }\DataTypeTok{x=}\NormalTok{clean_text, }\DataTypeTok{perl=}\OtherTok{TRUE}\NormalTok{)}
\NormalTok{   clean_text <-}\StringTok{ }\KeywordTok{gsub}\NormalTok{(}\DataTypeTok{pattern=}\NormalTok{extra_blank[}\DecValTok{1}\NormalTok{], }\DataTypeTok{replacement=}\NormalTok{extra_blank[}\DecValTok{2}\NormalTok{], }\DataTypeTok{x=}\NormalTok{clean_text, }\DataTypeTok{perl=}\OtherTok{TRUE}\NormalTok{)}
\NormalTok{   clean_text <-}\StringTok{ }\KeywordTok{gsub}\NormalTok{(}\DataTypeTok{pattern=}\NormalTok{arabic_numerals[}\DecValTok{1}\NormalTok{], }\DataTypeTok{replacement=}\NormalTok{arabic_numerals[}\DecValTok{2}\NormalTok{], }\DataTypeTok{x=}\NormalTok{clean_text, }\DataTypeTok{perl=}\OtherTok{TRUE}\NormalTok{)}
\NormalTok{   clean_text <-}\StringTok{ }\KeywordTok{gsub}\NormalTok{(}\DataTypeTok{pattern=}\NormalTok{multi_whitespace[}\DecValTok{1}\NormalTok{], }\DataTypeTok{replacement=}\NormalTok{multi_whitespace[}\DecValTok{2}\NormalTok{], }\DataTypeTok{x=}\NormalTok{clean_text, }\DataTypeTok{perl=}\OtherTok{TRUE}\NormalTok{)}
\NormalTok{   clean_text <-}\StringTok{ }\KeywordTok{gsub}\NormalTok{(}\DataTypeTok{pattern=}\NormalTok{whitespace_endline[}\DecValTok{1}\NormalTok{], }\DataTypeTok{replacement=}\NormalTok{whitespace_endline[}\DecValTok{2}\NormalTok{], }\DataTypeTok{x=}\NormalTok{clean_text, }\DataTypeTok{perl=}\OtherTok{TRUE}\NormalTok{)}
      \KeywordTok{return}\NormalTok{(clean_text)}
\NormalTok{\}}
\end{Highlighting}
\end{Shaded}

\texttt{Transcription} column of the first five inscriptions before
cleaning:

\begin{Shaded}
\begin{Highlighting}[]
\KeywordTok{print}\NormalTok{(EDH_tibble}\OperatorTok{$}\NormalTok{transcription[}\DecValTok{1}\OperatorTok{:}\DecValTok{5}\NormalTok{])}
\end{Highlighting}
\end{Shaded}

\begin{verbatim}
## Error in print(EDH_tibble$transcription[1:5]): object 'EDH_tibble' not found
\end{verbatim}

\texttt{Text\_cleaned} column, provided by EDH as a clean version of the
text, for comparison with the output of the
\texttt{cleaning\_intepretive\_edh} function:

\begin{Shaded}
\begin{Highlighting}[]
\KeywordTok{print}\NormalTok{(EDH_tibble}\OperatorTok{$}\NormalTok{text_cleaned[}\DecValTok{1}\OperatorTok{:}\DecValTok{5}\NormalTok{])}
\end{Highlighting}
\end{Shaded}

\begin{verbatim}
## Error in print(EDH_tibble$text_cleaned[1:5]): object 'EDH_tibble' not found
\end{verbatim}

Output of the \texttt{cleaning\_interpretive\_edh} function:

\begin{Shaded}
\begin{Highlighting}[]
\NormalTok{example_edh2 <-}\StringTok{ }\KeywordTok{as.data.frame}\NormalTok{(}\KeywordTok{cleaning_interpretive_edh}\NormalTok{(EDH_tibble}\OperatorTok{$}\NormalTok{transcription))}
\end{Highlighting}
\end{Shaded}

\begin{verbatim}
## Error in gsub(pattern = dubious_dot_subscript[1], replacement = dubious_dot_subscript[2], : object 'EDH_tibble' not found
\end{verbatim}

\begin{Shaded}
\begin{Highlighting}[]
\NormalTok{example_edh2[}\DecValTok{1}\OperatorTok{:}\DecValTok{5}\NormalTok{,]}
\end{Highlighting}
\end{Shaded}

\begin{verbatim}
## Error in eval(expr, envir, enclos): object 'example_edh2' not found
\end{verbatim}

\hypertarget{enriching-the-full-dataset-with-conservative-and-interpretive-cleaned-versions-of-the-text}{%
\subsubsection{Enriching the full dataset with conservative and
interpretive cleaned versions of the
text:}\label{enriching-the-full-dataset-with-conservative-and-interpretive-cleaned-versions-of-the-text}}

\begin{Shaded}
\begin{Highlighting}[]
\NormalTok{EDH_df_clean <-}\StringTok{ }\KeywordTok{as.data.frame}\NormalTok{(EDH_tibble) }\OperatorTok
\StringTok{  }\KeywordTok{mutate}\NormalTok{(}\DataTypeTok{clean_text_conservative =} \KeywordTok{cleaning_conservative_edh}\NormalTok{(EDH_tibble}\OperatorTok{$}\NormalTok{transcription)) }\OperatorTok
\StringTok{  }\KeywordTok{mutate}\NormalTok{(}\DataTypeTok{clean_text_interpretive =} \KeywordTok{cleaning_interpretive_edh}\NormalTok{(EDH_tibble}\OperatorTok{$}\NormalTok{transcription))  }
\end{Highlighting}
\end{Shaded}

\begin{verbatim}
## Error in as.data.frame(EDH_tibble): object 'EDH_tibble' not found
\end{verbatim}

\begin{center}\rule{0.5\linewidth}{0.5pt}\end{center}

\hypertarget{selecting-a-smaller-segment-for-testing}{%
\subsubsection{Selecting a smaller segment for
testing}\label{selecting-a-smaller-segment-for-testing}}

\begin{Shaded}
\begin{Highlighting}[]
\NormalTok{Thracia <-}\StringTok{ }\NormalTok{EDH_df_clean}\OperatorTok
\StringTok{  }\KeywordTok{filter}\NormalTok{(province_label }\OperatorTok{==}\StringTok{ "Thracia"}\OperatorTok{|}\StringTok{ }\NormalTok{province_label }\OperatorTok{==}\StringTok{ "Thracia?"}\NormalTok{)}
\end{Highlighting}
\end{Shaded}

\begin{verbatim}
## Error in eval(lhs, parent, parent): object 'EDH_df_clean' not found
\end{verbatim}

\hypertarget{comparing-new-cleaned-text-original-transcription-and-the-edh-cleaned-text-for-quality-of-cleaning}{%
\paragraph{Comparing new cleaned text, original transcription and the
EDH cleaned text for quality of
cleaning}\label{comparing-new-cleaned-text-original-transcription-and-the-edh-cleaned-text-for-quality-of-cleaning}}

\begin{Shaded}
\begin{Highlighting}[]
\NormalTok{number <-}\StringTok{ }\DecValTok{24}
\KeywordTok{print}\NormalTok{(Thracia}\OperatorTok{$}\NormalTok{clean_text_interpretive[number]) }\CommentTok{# output of cleaning_interpretive function}
\end{Highlighting}
\end{Shaded}

\begin{verbatim}
## Error in print(Thracia$clean_text_interpretive[number]): object 'Thracia' not found
\end{verbatim}

\begin{Shaded}
\begin{Highlighting}[]
\KeywordTok{print}\NormalTok{(Thracia}\OperatorTok{$}\NormalTok{transcription[number])           }\CommentTok{# original text to be cleaned  }
\end{Highlighting}
\end{Shaded}

\begin{verbatim}
## Error in print(Thracia$transcription[number]): object 'Thracia' not found
\end{verbatim}

\begin{Shaded}
\begin{Highlighting}[]
\KeywordTok{print}\NormalTok{(Thracia}\OperatorTok{$}\NormalTok{text_cleaned[number])            }\CommentTok{# text_cleaned provided by EDH}
\end{Highlighting}
\end{Shaded}

\begin{verbatim}
## Error in print(Thracia$text_cleaned[number]): object 'Thracia' not found
\end{verbatim}

\hypertarget{saving-as-json-to-local-space}{%
\subsection{Saving as JSON to local
space}\label{saving-as-json-to-local-space}}

\begin{Shaded}
\begin{Highlighting}[]
\NormalTok{Thracia_json <-}\StringTok{ }\KeywordTok{toJSON}\NormalTok{(Thracia)}
\end{Highlighting}
\end{Shaded}

\begin{verbatim}
## Error in toJSON(Thracia): object 'Thracia' not found
\end{verbatim}

\begin{Shaded}
\begin{Highlighting}[]
\CommentTok{#write(Thracia_json, "../../outputs/EDH_Thracia.json")}
\end{Highlighting}
\end{Shaded}

\hypertarget{writing-as-json-to-sciencedata.dk---to-be-implemented}{%
\subsection{Writing as JSON to Sciencedata.dk - to be
implemented}\label{writing-as-json-to-sciencedata.dk---to-be-implemented}}

\begin{Shaded}
\begin{Highlighting}[]
\NormalTok{EDH_clean_text_json <-}\StringTok{ }\KeywordTok{toJSON}\NormalTok{(EDH_df_clean)}
\end{Highlighting}
\end{Shaded}

\begin{verbatim}
## Error in toJSON(EDH_df_clean): object 'EDH_df_clean' not found
\end{verbatim}

\begin{Shaded}
\begin{Highlighting}[]
\KeywordTok{write}\NormalTok{(EDH_clean_text_json, }\DataTypeTok{file=}\StringTok{"EDH_clean_text_sample.json"}\NormalTok{)}
\end{Highlighting}
\end{Shaded}

\begin{verbatim}
## Error in cat(x, file = file, sep = c(rep.int(sep, ncolumns - 1), "\n"), : object 'EDH_clean_text_json' not found
\end{verbatim}

\begin{Shaded}
\begin{Highlighting}[]
\NormalTok{user <-}\StringTok{ }\KeywordTok{readline}\NormalTok{(}\StringTok{"your sciencedata username: "}\NormalTok{)}
\end{Highlighting}
\end{Shaded}

\begin{verbatim}
## your sciencedata username:
\end{verbatim}

\begin{Shaded}
\begin{Highlighting}[]
\KeywordTok{request}\NormalTok{(}\StringTok{"EDH_clean_text_sample.json"}\NormalTok{, }\DataTypeTok{path=}\StringTok{"/sharingout/648597@au.dk/SDAM_root/SDAM_data/EDH/"}\NormalTok{, }
        \DataTypeTok{method=}\StringTok{"PUT"}\NormalTok{, }\DataTypeTok{cred=}\KeywordTok{c}\NormalTok{(user, }\KeywordTok{getPass}\NormalTok{(}\StringTok{"your sciencedata password: "}\NormalTok{))) }
\end{Highlighting}
\end{Shaded}

\begin{verbatim}
## Please enter password in TK window (Alt+Tab)
\end{verbatim}

\begin{verbatim}
## Response [https://sciencedata.dk/sharingout/648597@au.dk/SDAM_root/SDAM_data/EDH//EDH_clean_text_sample.json]
##   Date: 2020-05-13 09:49
##   Status: 401
##   Content-Type: text/html; charset=UTF-8
## <EMPTY BODY>
\end{verbatim}

\end{document}
